\subsection{Normed Space. Banach Space.}

\begin{question}
    Show that the norm $\norm{x}$ of $x$ is the distance of $x$ from 0.
    \label{section2.2-1}
\end{question}
\begin{proof}
    By the definition of a norm, we have
    \[d(x,y) = \norm{x-y}.\]
    Subsitituting $y = 0$, we get
    \[\norm{x} = d(x,0).\]
\end{proof}

\begin{question}
    Verify that the usial length of a vector in the plane or in three- dimensional space has the properties (N1) to (N4).
    \label{section2.2-2}
\end{question}
\begin{proof}
    The properties referred to are:
    \begin{enumerate}
        \item $\norm{x} \geq 0$.
        \item $\norm{x} = 0 \iff x = 0$
        \item $\norm{\alpha x} = \abs{\alpha} \norm{x}.$
        \item $\norm{x + y} \leq \norm{x} + \norm{y}.$
    \end{enumerate}
    Now, the length of vector $x = (a,b,c)$ in three dimensions is given by 
    \[\norm{x} = \sqrt{a^2 + b^2 + c^2}.\]
    Clearly, this value is always non-negative. Also, 
    \[\sqrt{a^2 + b^2 + c^2} = 0 \implies a = b = c = 0.\]
    Similarly, $a = b = c$ gives us $\norm{x} = 0$. Finally, 
    \[\norm{\alpha x} = \sqrt{\alpha^2 a^2 + \alpha^2 b^2 + \alpha^2 c^2} = \sqrt{\alpha^2} \sqrt{a^2 + b^2 + c^2} = \abs{\alpha}\norm{x}.\]
    We now show that the triangle inequality also holds. Let $x = (a_1 , b_1 , c_1)$ and $y = (a_2 , b_2 , c_2)$. Then,
    \[\norm{x+y} = \sqrt{(a_1 + a_2)^2 + (b_1 + b_2)^2 + (c_1 + c_2)^2}.\]
    Recall Hölder's inequality for $a_i , b_i \geq 0$:
    \[\sum_{i=1}^n a_ib_i \leq \left(\sum_{i=1}^n a_i^p\right)^{\frac{1}{p}} \left(\sum_{i=1}^n b_i^q\right)^{\frac{1}{q}},\]
    where $p^{-1} + q^{-1} = 1$. Using this, we get
    \[a_1a_2 + b_1b_2 + c_1c_2 \leq \sqrt{a_1^2 + b_1^2 + c_1^2}\sqrt{a_2^2 + b_2^2 + c_2^2}.\]
     and hence,
     \[\norm{x+y} \leq \sqrt{a_1^2 + a_2^2 + b_1^2 + b_2^2 + c_1^2 + c_2^2 + 2 \sqrt{a_1^2 + b_1^2 + c_1^2}\sqrt{a_2^2 + b_2^2 + c_2^2}} = \sqrt{a_1^2 + b_1^2 + c_1^2} + \sqrt{a_2^2 + b_2^2 + c_2^2},\]
     and thus, 
     \[\norm{x+y} \leq \norm{x} + \norm{y}.\]
\end{proof}

\begin{question}
    Prove (2).
    \label{section2.2-3}
\end{question}
\begin{proof}
    We wish to show
    \[\abs{\norm{x} - \norm{y}} \leq \norm{x - y}.\]
    Using the triangle inequality, we have
    \[\norm{x} = \norm{(x-y) + y} \leq \norm{x-y} + \norm{y} \implies \norm{x} - \norm{y} \leq \norm{x-y}.\]
    Similarly, 
    \[\norm{y} - \norm{x} \leq \norm{y-x} = \norm{x-y}\],
    and hence, combining both inequalities results in 
    \[\abs{\norm{x} - \norm{y}} \leq \norm{x-y}.\]
\end{proof}

\begin{question}
    Show that we may replace (N2) by
    \[\norm{x} = 0 \implies x = 0\]
    without altering the concept of a norm. Show that non-negativity of a norm follows from (N3) and (N4).
    \label{section2.2-4}
\end{question}
\begin{proof}
    First, we wish to show that we can replace 
    \[\norm{x} = 0 \iff x = 0\]
    by 
    \[\norm{x} = 0 \implies x = 0.\]
    This means, we wish to show that the condition 
    \[x = 0 \implies \norm{x} = 0\]
    is redundant. Since we have already shown that $\norm{x} = d(x,0)$, substituting $x = 0$ gives us the claim. Hence, this condition follows from the definition of a metric induced by a norm, and is hence, redundant. Another way to see this is: we know that $\norm{\alpha x} = \abs{\alpha}\norm{x}$. Substituting $\alpha = 0$ gives us
    \[\norm{0} = 0.\]

    We now wish to show that we can prove the non-negativity of a norm using the following conditions:
    \[\norm{\alpha x} = \abs{\alpha}\norm{x}\]
    \[\norm{x+y} \leq \norm{x} + \norm{y}.\]
    We have that 
    \[0 = \norm{x - x} \leq \norm{x} + \abs{-1}\norm{x} = 2\norm{x}\]
    and hence, $\norm{x} \geq 0$.
\end{proof}

\begin{question}
    Show that (3) defines a norm.
    \label{section2.2-5}
\end{question}
\begin{proof}
    We wish to show that
    \[\norm{x} = \left(\sum_{j=1}^n \abs{\xi_j}^2 \right)^{1/2} = \sqrt{\abs{\xi_1}^2 + \ldots + \abs{\xi_n}^2}\]
    is a norm. We have already shown this in \ref{section2.2-2} for $n = 3$, and hence, the proof follows in exactly the same manner. We demonstrate the proof for trinagle inequality once again. Let $x = (\xi_1 , \ldots , \xi_n)$ and $y = (\eta_1 , \ldots , \eta_n)$. Then,
    \begin{align*}
        \norm{x + y} &= \sqrt{\sum_{i=1}^n (\xi_i + \eta_i)^2}
        \\
        &= \sqrt{\sum_{i=1}^n \xi_i^2 + \sum_{i=1}^n \eta_i^2 + 2 \sum_{i=1}^n \xi_i\eta_i}
        \\
        &\leq \sqrt{\sum_{i=1}^n \xi_i^2 + \sum_{i=1}^n \eta_i^2 + 2 \sqrt{\sum_{i=1}^n \xi_i^2}\sqrt{\sum_{i=1}^n \eta_i^2}}
        \\
        &= \sqrt{\sum_{i=1}^n \xi_i^2} + \sqrt{\sum_{i=1}^n \eta_i^2}
        \\
        &= \norm{x} + \norm{y}
    \end{align*}
    where the inequality makes use of Hölder's inequality.
    
\end{proof}

\begin{question}
    Let $X$ be the vector space of all ordered pairs $x = (\xi_1 , \xi_2)$, $y = (\eta_1 , \eta_2)$ of real numbers. Show that norms on $X$ are defined by
    \[\norm{x}_1 = \abs{\xi_1} + \abs{\xi_2}\]
    \[\norm{x}_2 = \sqrt{\abs{\xi_1}^2 + \abs{\xi_2}^2}\]
    \[\norm{x}_\infty = \max\{\abs{\xi_1} , \abs{\xi_2}\}.\]
    \label{section2.2-6}
\end{question}
\begin{proof}
    We first show $\norm{x}_1$ is a valid norm. Clearly, it is always non-negative, and since it is the sum of non-negative terms, it is zero iff each of the terms is zero, resulting in the condition $\norm{x}_1 = 0 \iff x = 0$. Now, we have
    \[\norm{\alpha x}_1 = \abs{\alpha \xi_1} + \abs{\alpha \xi_2} = \abs{\alpha} \abs{\xi_1} + \abs{\alpha}\abs{xi_2} = \abs{\alpha} \norm{x}_{1}.\]
    Also, using the triangle inequality for the absolute values on real numbers gives:
    \[\norm{x + y} = \abs{\xi_1 + \eta_1} + \abs{\xi_2 + \eta_2} \leq \abs{\xi_1} + \abs{\xi_2} + \abs{\eta_1} + \abs{\eta_2} = \norm{x}_1 + \norm{x}_2.\]
    Thus, $\norm{x}_1$ is a valid norm.

    We have shown $\norm{x}_2$ is a valid norm for three-dimensional points in \ref{section2.2-3} and for $n-$dimensional points in \ref{section2.2-5}.

    We now show $\norm{x}_\infty$ is also a valid norm. First, by definition it is always non-negative, and the maximum of two non-negative terms can be zero iff each of the terms is zero, resulting in the condition $\norm{x}_\infty = 0 \iff x = 0$. Further,
    \[\norm{\alpha x}_\infty = \max\{\abs{\alpha \xi_1} , \abs{\alpha \xi_2}\} = \abs{\alpha}\max\{\abs{\xi_1} , \abs{\xi_2}\} = \abs{\alpha} \norm{x}_\infty.\]
    Finally, the triangle inequality follows since:
    \[\norm{x + y}_\infty = \max\{\abs{\xi_1 + \eta_1} , \abs{\xi_2 + \eta_2}\} = \abs{\xi_i + \eta_i} \leq \abs{\xi_i} + \abs{\eta_i} \leq \max\{\abs{\xi_1} , \abs{\xi_2}\} + \max\{\abs{\eta_1} , \abs{\eta_2}\} = \norm{x}_\infty + \norm{y}_\infty\]
    Thus, $\norm{x}_\infty$ is a valid norm.
\end{proof}

\begin{question}
    Verify that (4) satisfies (N1) to (N4).
    \label{section2.2-7}
\end{question}
\begin{proof}
    We wish to show that the $\ell_p$ norm, defined as 
    \[\norm{x}_p = \left(\sum_{j=1}^n \abs{\xi_j}^p \right)^{1/p}\]
    is a valid norm. First, it is easy to see that it is the sum of non-negative terms, and hence, is always non-negative. Also, it would only be zero if each of the terms is zero, resulting in the condition that $\norm{x}_p = 0 \iff x = 0$. Further,
    \[\norm{\alpha x}_p = \left(\sum_{j=1}^n \abs{\alpha \xi_j}^p \right)^{1/p} = \left(\abs{\alpha}^p\sum_{j=1}^n \abs{\xi_j}^p \right)^{1/p} = \abs{\alpha} \left(\sum_{j=1}^n \abs{\xi_j}^p \right)^{1/p} = \abs{\alpha} \norm{x}_p.\]
    Finally, the triangle inequality follows from Minkowski's inequality:
    \begin{align*}
        \norm{x + y}_p = \left(\sum_{j=1}^n \abs{\xi_j + \eta_j}^p \right)^{1/p} \leq \left(\sum_{j=1}^n \abs{\xi_j}^p \right)^{1/p} + \left(\sum_{j=1}^n \abs{\eta_j}^p \right)^{1/p} = \norm{x}_p + \norm{y}_p
    \end{align*}
\end{proof}


\begin{question}
    Show that these norms on the vector space of ordered $n-$tuples of numbers are valid norms.
    \[\norm{x}_1 = \abs{\xi_1} + \ldots + \abs{\xi_n}\]
    \[\norm{x}_2 = \sqrt{\abs{\xi_1}^2 + \ldots + \abs{\xi_n}^2}\]
    \[\norm{x}_\infty = \max\{\abs{\xi_1} , \ldots , \abs{\xi_n}\}\]
    \label{section2.2-8}
\end{question}
\begin{proof}
    We first begin with $\norm{x}_1$, which is clearly non-negative and can only be zero iff $x  = 0$. Also, using the fact that $\abs{\alpha x} = \abs{\alpha}\abs{x}$, we have
    \[\norm{\alpha x}_1 = \sum_{i=1}^n \abs{\alpha \xi_i} = \sum_{i=1}^n \abs{\alpha}\abs{\xi_i} = \abs{\alpha}\norm{x}_1.\]
    Finally, the triangle inequality follows from the triangle inequality for real numbers:
    \[\norm{x + y}_1 = \sum_{i=1}^n \abs{\xi_i + \eta_i} \leq \sum_{i=1}^n \abs{\xi_i} + \sum_{i=1}^n\abs{\eta_i} = \norm{x}_1 + \norm{y}_1.\]
    Thus, $\norm{x}_1$ is a valid norm.

    We have already shown $\norm{x}_2$ is a valid norm in \ref{section2.2-5}.

    We now show that $\norm{x}_\infty$ is a valid norm. Clearly, it is non-negative, and the maximum of non-negative terms can be zero iff each of the terms is zero , thus resulting in the condition $\norm{x} = 0 \iff x = 0$. Further,
    \[\norm{\alpha x}_\infty = \max_{i \in [n]} \abs{\alpha \xi_i} = \abs{\alpha} \max_{i \in [n]} \abs{\xi_i} = \abs{\alpha} \norm{x}_\infty.\]
    The triangle inequality follows since:
    \[\norm{x + y}_\infty = \max_{i \in [n]} \abs{\xi_i + \eta_i} = \abs{\xi_j + \eta_j} \leq \abs{\xi_j} + \abs{\eta_j} \leq \max_{i \in [n]} \abs{\xi_i} + \max_{i \in [n]} \abs{\eta_i} = \norm{x}_\infty + \norm{y}_\infty.\]
    Thus, $\norm{x}_\infty$ is a valid norm.
\end{proof}


\begin{question}
    Verify that (5) defines a norm.
    \label{section2.2-9}
\end{question}
\begin{proof}
    We wish to show that 
    \[\norm{x} = \max_{t \in J} \abs{x(t)}\]
    defines a norm. By definition, this is non-negative, and the only way $\norm{x} = 0$ is if for each $t \in J, x(t) = 0$, or in other words, $x = 0$. Further, 
    \[\norm{\alpha x} = \max_{t \in J} \abs{\alpha x(t)} = \abs{\alpha} \max_{t \in J} \abs{x(t)} = \abs{\alpha} \norm{x}.\]
    The triangle inequality follows since:
    \[\norm{x + y} = \max_{t \in J} \abs{x(t) + y(t)} \leq \abs{x(t_0) + y(t_0)} \leq \abs{x(t_0)} + \abs{y(t_0)} \leq \max_{t \in J} \abs{x(t)} + \max_{t \in J} \abs{y(t)} = \norm{x} + \norm{y}.\]
    Thus, it is a valid norm.
\end{proof}

\begin{question}
    The sphere
    \[S(0;1) = \{x \in X \mid \norm{x} = 1\}\]
    in a normed space $X$ is called the unit sphere. Show that for the norms in \ref{section2.2-6} and for the norm defined by 
    \[\norm{x}_4 = (\xi_1^4 + \xi_2^4)^{1/4}\]
    The unit spheres look as shown in Fig.16.
    \label{section2.2-10}
\end{question}
\begin{proof}
    It is not clear on how to show this. We will attempt to show it in the following way: $\norm{x}_1$, $\norm{x}_2$, and $\norm{x}_\infty$ in $\R^2$ depict a diamond, circle, and square, respectively. Then, we wish to show that if $x_1$ satisfies $\norm{x_1}_2 = 1$, $x_2$ satisfies $\norm{x_2}_4 = 1$, and  $x_3$ satisfies $\norm{x_3}_\infty = 1$, then ,
    \[\norm{x_1}_2 \leq \norm{x_2}_2 \leq \norm{x_3}_\infty\],
    or in other words, the distance from the origin of these points increases.

    First, 
    \[\norm{x}_1 = 1 \implies \abs{\xi_1} + \abs{\xi_2} = 1.\]
    This results in the following $4$ lines: $\pm \xi_1 \pm \xi_2 = 1$, which results in the diamond. Next, 
    \[\norm{x}_2 = 1 \implies \xi_1^2 + \xi_2^2 = 1\]
    results in the equation of a circle centered around the origin with radius 1. Finally, 
    \[\norm{x}_\infty = 1 \implies \max\{\abs{\xi_1} , \abs{\xi_2}\} = 1\]
    results in a square.
    Now, for $\norm{x_3}_\infty = 1$, anuy point of the form $(\pm 1  , t)$ or $(t , \pm 1)$ works, where $t \ in [-1,1]$ and thus, $\norm{x_3}_2 \leq \sqrt{1 + 1} = \sqrt{2}$ and $\norm{x_3}_2 \geq \sqrt{1} = 1$. Finally, we see that $x_2 = (\sqrt{\cos \theta} , \sqrt{\sin \theta})$ satisfies $\norm{x_2}_4 = 1$, and hence, 
    \[\norm{x_2}_2 = \sqrt{\cos \theta + \sin \theta} = \sqrt{\sqrt{2} \sin \left(\frac{\pi}{4} + \theta \right)} \leq 2^{1/4}\]
    and hence. we have
    \[\norm{x_1}_2 \leq \norm{x_2}_2 \leq \norm{x_3}_\infty\],
    which shows that the distance of points which satisfy $\norm{.}_4 = 1$ is more than that of points which satisfy $\norm{.}_2 = 1$ but less than that of $\norm{.}_\infty = 1$.
\end{proof}

\begin{question}
    A subset $A$ of vector space $X$ is said to be convex if $x , y \in A$ implies
    \[M = \{z \in X \mid z = \alpha x + (1-\alpha)y , 0 \leq \alpha \leq 1\} \subset A.\]
    $M$ is called a closed segment with boundary points $x$ and $y$, any other $z \in M$ is called an interior point of $M$. Show that the closed unit ball
    \[\tilde{B}(0,1) = \{x \in X \mid \norm{x} \leq 1\}\]
    in a normed space is convex.
    \label{section2.2-11}
\end{question}
\begin{proof}
    Choose two points $x , y \in \tilde{B}(0,1)$ such that $\norm{x} \leq 1$ and $\norm{y} \leq 1$. Then, the norm of any convex combination of $x$ and $y$ is given by
    \[\norm{\alpha x + (1-\alpha)y} \leq \abs{\alpha} \norm{x} + \abs{1-\alpha} \norm{y} \leq 1\]
    which shows that any convex combination of the points lies inside the closed ball. Here, the first inequality follows by using the triangle inequality and scalar multiplication property of the norm in tandem.
\end{proof}

\begin{question}
    Using \ref{section2.2-11}, show that
    \[\varphi(x) = \left( \sqrt{\abs{\xi_1}} + \sqrt{\abs{\xi_2}} \right)^2\]
    does not define a norm on the vector space of all ordered pairs of real numbers.
    \label{section2.2-12}
\end{question}
\begin{proof}
    In \ref{section2.2-11} we have shown that the closed ball of radius 1 is a convex set with respect to any norm. Suppose, $\varphi(x)$ defines a norm such that $\varphi(x) \leq 1$ and $\varphi(y) \leq 1$. Then, we have
    \begin{align*}
        \varphi(\alpha x + (1-\alpha)y) &= \left( \sqrt{\abs{\alpha \xi_1 + (1-\alpha) \eta_1}} + \sqrt{ \abs{\alpha \xi_2 + (1-\alpha) \eta_2} } \right)^2
        \\
        &\leq \left( \sqrt{\alpha \abs{\xi_1} + (1-\alpha) \abs{\eta_1}} + \sqrt{\alpha \abs{\xi_2} + (1-\alpha) \abs{\eta_2}} \right)^2
        \\
        &= \alpha (\abs{\xi_1} + \abs{\xi_2}) + (1-\alpha) (\abs{\eta_1} + \abs{\eta_2}) + 2\sqrt{(\alpha \abs{\xi_1} + (1-\alpha) \abs{\eta_1})(\alpha \abs{\xi_2} + (1-\alpha) \abs{\eta_2})}
    \end{align*}

    Now, since $\phi(x) = (\abs{\xi_1} + \abs{\xi_2}) + s \sqrt{\abs{\xi_1\xi_2}} \leq 1$, we have that $(\abs{\xi_1} + \abs{\xi_2}) \leq 1$. Similarly, $(\abs{\eta_1} + \abs{\eta_2}) \leq 1$ and hence, we get
    \[\varphi(\alpha x + (1-\alpha)y) \leq 1 + 2\sqrt{(\alpha \abs{\xi_1} + (1-\alpha) \abs{\eta_1})(\alpha \abs{\xi_2} + (1-\alpha) \abs{\eta_2})}\]
    Since each of the terms inside the square root are non-negative, we have that the the additional quantity is non-negative, and hence, $\varphi(\alpha x + (1-\alpha)y)$ may be greater than $1$, leading to the fact that the subset is no longer convex.
\end{proof}

\begin{question}
    Show that the discrete metric on a vector space $X \neq \{0\}$ cannot be obtained from a norm.
    \label{section2.2-13}
\end{question}
\begin{proof}
    Let $d$ be a metric induced by a norm. Then, $d$ satisfies the following:
    \[d(x+a, y+a) = \norm{x-y} = d(x,y), \]
    \[d(\alpha x , \alpha y) = \norm{\alpha x - \alpha y} = \abs{\alpha} \norm{x - y} = \abs{\alpha} d(x,y).\]
    Suppose $x \neq y$. Then, $d(x,y) = 1$. However, for $\alpha \neq \pm 1$, $d(x,y) \neq d(\alpha x , \alpha y)$, and hence, the metric is not induced from a norm.
\end{proof}

\begin{question}
    If $d$ is a metric on a vector space $X \neq \{0\}$ obtained from a norm, and $\tilde{d}$ is defined by
    \[\tilde{d}(x,x) = 0 \;,\; \tilde{d}(x,y) = d(x,y) + 1 \text{ for } x\neq y.\]
    Show that $\tilde{d}$ cannot be obtained from a norm.
    \label{section2.2-14}
\end{question}

\begin{proof}
    Suppose $\tilde{d}$ was induced from a norm. Then, 
    \[\tilde{d}(\alpha x , \alpha y) = d(\alpha x , \alpha y) + 1 = \abs{\alpha} d(x , y) + 1.\]
    However, we also have
    \[\tilde{d}(\alpha x , \alpha y) = \abs{\alpha} \tilde{d}(x , y) = \abs{\alpha} d(x,y) + \abs{\alpha}.\]
    Thus, if $\alpha \neq \pm 1$, then, $\tilde{d}$ is not a metric induced by a norm. Since, this should hold for all values of $\alpha$, $\tilde{d}$ is not a metric induced by a norm.
\end{proof}

\begin{question}
    Show that a subset $M$ in a normed space $X$ is bounded if and only if there is a positive number $c$ such that $\norm{x} \leq c$ for every $x \in M$.
    \label{section2.2-15}
\end{question}
\begin{proof}
    Recall that a subset $M$ is bounded if 
    \[\delta(M) = \sup_{x , y \in M} d(x,y) \l \infty.\]

    Suppose the set is bounded. Then, $\sup_{M \times M} d(x,y) \l \infty$. Let $d(x,y) \l c$ for all pairs $(x,y) \in M\times M$. Then, 
    \[\norm{x} \leq \norm{x-y} + \norm{y} \leq c + \norm{y}\]
    which is bounded above by some other constant $c^\prime.$

    On the other hand, we have
    \[d(x,y) = \norm{x-y} \leq \norm{x} + \norm{y} \leq 2c\]
    and hence, 
    \[\sup_{(x,y) \in M\times M} d(x,y) = 2c.\]
    This finishes the proof.
\end{proof}