\subsection{Compactness and Finite Dimension}

\begin{question}
    Show that $\R^n$ and $\C^n$ are not compact.
    \label{section2.5-1}
\end{question}
\begin{proof}
    First, note that $\R^n$ and $\C^n$ are both $n-$dimensional, i.e, finite dimensional. We know that a finite dimensional subspace is compact if and only if it is closed and bounded. However, $\R^n$ and $\C^n$ are not bounded, and hence, not compact.
\end{proof}

\begin{question}
    Show that a discrete metric space $X$ consisting of infinitely many points is not compact.
    \label{section2.5-2}
\end{question}
\begin{proof}
    Consider any sequence $(x_n)$ with all elements distinct. Then, we have that for all $m,n$
    \[d(x_m , x_n) = 1.\]
    Thus, this sequence can never converge, and neither can any subsequence. What adds to this is the fact that $X$ has infinite points, which means that none of the elements or subsequences would repeat. Thus, the space is not compact.
\end{proof}

\begin{question}
    Give examples of compact and non-compact curves in the plane $\R^2$.
    \label{section2.5-3}
\end{question}
\begin{proof}
    Any closed and bounded curve would be compact, such as the closed ball of radius $r$. The open ball of radius $r$ is open and hence, not compact.
\end{proof}

\begin{question}
    Show that for an infinite subset $M$ in the space $s$ to be compact, it is necessary that there are numbers $\gamma_1 , \gamma_2 , \ldots$ such that for all $x = (\xi_k(x)) \in M$ we have $\abs{\xi_k(x)} \leq \gamma_k$.
    \label{section2.5-4}
\end{question}
\begin{proof}
    Recall that $s$ is the space of all bounded or unbounded sequences. The metric defined on this space is 
    \[d(x,y) = \sum_{j=1}^\infty \frac{1}{2^j} \frac{\abs{\xi_j - \eta_j}}{1 + \abs{\xi_j + \eta_j}}.\]
    We can show this space is complete as follows: Consider $(x_n)_n = ((\xi^{(n)}_j)_j)_n$ to be a sequence that is Cauchy, i.e
    \[d(x_m , x_n) = \sum_{j=1}^\infty \frac{1}{2^j} \frac{\abs{\xi_j - \eta_j}}{1 + \abs{\xi_j + \eta_j}} \l \epsilon \implies \max_{j} \frac{\abs{\xi^{(n)}_j - \xi^{(m)}_j}}{1 + \abs{\xi^{(n)}_j - \xi^{(m)}_j}} \l 2\epsilon \implies \max_{j} \abs{\xi^{(n)}_j - \xi^{(m)}_j} \l \frac{2\epsilon}{1 - 2\epsilon}.\]
    Thus, for a fixed $j$, the sequence $(\xi^{(1)}_j , \xi^{(2)}_j, \ldots, $ is Cauchy, and hence, converges to $\xi_j$. Thus, considering $x = (\xi_1 , \xi_2 , \ldots) \in M$, we have that $x_n \rightarrow x$. Hence, this space is complete. For $s$ to be compact, we require it to be bounded. This means that bounding the absolute value of each of the terms for all sequences is sufficient.
\end{proof}

\begin{question}
    A metric space $X$ is said to be locally compact if every point of $X$ has a neighborhood that is compact. Show that $\R^n$ and $\C^n$ is locally compact.
    \label{section2.5-5}
\end{question}
\begin{proof}
    Take any point $x \in \R^n$. We know that any neighborhood of $x$ contains an $\epsilon-$ball around $x$. Let the neighborhood be the closed ball of radius $\epsilon$. Since,
    \[B(x , \epsilon) \subset \tilde{B}(x , \epsilon),\]
    this closed ball is a valid neighborhood. Now, this neighborhood is bounded (since we can bound $\tilde{B}(x , \epsilon)$ with another open ball of radius $\epsilon + \delta, \delta \g 0$) and is closed, and hence, is compact. Thus, $\R^n$ is locally compact.
\end{proof}

\begin{question}
    Show that a compact metric space $X$ is locally compact.
    \label{section2.5-6}
\end{question}
\begin{proof}
    Consider some point $x \in X$. Since $X$ belongs to set of all open sets in $X$, there exists a ball around $x$ completely contained in $X$ and hence, $X$ is a neighborhood itself. Since $X$ is compact, it follows that $X$ is locally compact.
\end{proof}

\begin{question}
    If $\dim Y \l \infty$ in Reisz Lemma, show that one can choose $\theta = 1$.
    \label{section2.5-7}
\end{question}
\begin{proof}
    \textcolor{red}{TODO}.
\end{proof}

\begin{question}
    In \ref{section2.4-7}, section 2.4, show that there is an $a \g 0$ such that $a \norm{x}_2 \l \norm{x}.$
    \label{section2.5-8}
\end{question}
\begin{proof}
    \textcolor{red}{TODO}
\end{proof}

\begin{question}
    If $X$ is a compact metric space and $M \subset X$ is closed, show that $M$ is compact.
    \label{section2.5-9}
\end{question}
\begin{proof}
    Let $(x_n)$ be any sequence in $M$ that converges to $x$. Since $M$ is closed, $x \in M$. Also, since $X$ is compact, we have that atleast one subsequence of $(x_n)$ converges. Since $x_n \rightarrow x$, the subsequence also converges to $x$, and $x \in M$. Thus, we have a sequence in $M$ that has a subsequence that converges to some point in $M$. Hence, $M$ is compact.
\end{proof}

\begin{question}
    Let $X$ and $Y$ be metric spaces, $X$ is compact, and $T : X \mapsto Y$ is bijective and continuous. Show that $T$ is a homomorphism.
    \label{section2.5-10}
\end{question}
\begin{proof}
    Recall the definition of a homomorphism: a continuous and bijective mapping whose inverse is also continuous.

    Now, we wish to show that $T^{-1} : Y \mapsto X$ is continuous given that $X$ and $Y$ are compact. Let $(y_n)$ be some sequence in $M \subset Y$ that has a convergent subsequence $(y_{n_k})$ that converges to $y$. Now, consider the sequence $(x_n) = (T^{-1}y_n)$. Since $X$ is compact, there exists some subsequence $(x_{n_j})$ that converges to $x_0$. Also, assume $Tx = y$.
    Now, since $Tx_n = y_n$, using the continuity of $T$, we have
    \[y_n = Tx_n \rightarrow Tx_0 \]
    since a sequence converges to the same limit as that of its subsequences. Now, since the limits are unique, we have that $Tx_0 = y$. But since $T$ is bijective, and hence, one-one, $x_0 = x$. Thus, we have that under the mapping $T^{-1}$, a sequence in $Y$ and it's corresponding image in $X$ are convergent, and hence, $T^{-1}$ is continuous.
\end{proof}