\subsection{Further Examples of Metric Spaces}

\begin{question}
    Show that in 1.2-1, we can obtain another metric by replacing $1/2^i$ with $\mu_i$ such that $\sum \mu_i$ converges.
    \label{section1.2-1}
\end{question}
\begin{proof}
    We wish to show that for some $\mu_j \g 0$ such that $\sum mu_j$ converges, and two sequences $x = (\xi_i)$ and $y = (\eta_i)$, we have
    \[d(x,y) = \sum_{j=1}^\infty \mu_j \frac{\abs{\xi_j - \eta_j}}{1 + \abs{\xi_j - \eta_j}}  \]
    is a valid metric defined on the sequence space.

    The proof follows on the same lines as the one given for 1.2-1. It is clear that $d(x,y)$ is non-negative, symmetric, and satisfies $d(x,y) = 0 \iff x = y$. However, we wish to show this is bounded, i.e
    \[d(x,y) = \sum_{j=1}^\infty \mu_j \frac{\abs{\xi_j - \eta_j}}{1 + \abs{\xi_j - \eta_j}} \leq \sup_{j \in \N} \frac{\abs{\xi_j - \eta_j}}{1 + \abs{\xi_j - \eta_j}} \sum_{j=1}^\infty \mu_j\]
    Note that the RHS is bounded if and only if $\sum \mu_j$ is convergent (since the supremum term can be upper bounded by 1). Further, the proof for the triangle inequality follows in the same fashion, define $f(t) = \frac{t}{1 + t}$. Then, $f(t)$ is monotonically increasing, and hence, using the triangle inequality for numbers, we get
    \[f(\abs{a+b}) \leq f(\abs{a} + \abs{b})\]
    and hence, substituting $a = \xi_j - \zeta_j$ and $b = \zeta_j - \eta_j$, we get
    \[\frac{\abs{\xi_j - \eta_j}}{1 + \abs{\xi_j - \eta_j}} \leq \frac{\abs{\xi_j - \zeta_j} + \abs{\zeta_j - \eta_j}}{1 + \abs{\xi_j - \zeta_j} + \abs{\zeta_j - \eta_j}} \leq \frac{\abs{\xi_j - \zeta_j}}{1 + \abs{\xi_j - \zeta_j}} + \frac{\abs{\zeta_j - \eta_j}}{1 +  + \abs{\zeta_j - \eta_j}}\]
    Multiplying by $\mu_j$ and summing over all $j \in \N$ finishes the proof.
\end{proof}

\begin{question}
    Using (6) show that the geometric mean of two positive numbers does not exceed the arithmetic mean.
    \label{section1.2-2}
\end{question}
\begin{proof}
    We wish to show that the geometric mean does not exceed the arithmetic mean using the equation:
    \[xy \leq \frac{x^p}{p} + \frac{y^q}{q}\]
    where $p^{-1} + q^{-1} = 1$.
    
    Substituting $x = \sqrt{a} , y = \sqrt{b} , p = q = 2$, we get
    \[\sqrt{ab} \leq \frac{a + b}{2}.\]
\end{proof}

\begin{question}
    Show that the Cauchy Schwarz inequality implies
    \[\left(\sum_{i=1}^n \abs{\xi_i} \right)^2 \leq n \sum_{i=1}^n \abs{\xi_i}^2\]
    \label{section1.2-3}
\end{question}
\begin{proof}
    The Cauchy-Schwarz inequality states that
    \[\left(\sum_{i=1}^n \abs{\xi_i} \abs{\eta_i} \right)^2 \leq  \left( \sum_{i=1}^n \abs{\eta_i}^2 \right)\left( \sum_{i=1}^n \abs{\xi_i}^2 \right)\]
    Putting $\eta_i = 1$ for all $i \in [n]$ completes the proof.
\end{proof}

\begin{question}
    Find a sequence which converges to $0$, but is not in any space $\ell^p$, $1 \leq p \l \infty$. 
    \label{section1.2-4}
\end{question}
\begin{proof}
    Recall that the $\ell^p$ space consists of sequences $x = (\xi_i)$ such that $\sum \abs{\xi_i}^p$ converges. The sequence $\xi_i = \frac{1}{\log(i + 1)}$ converges to 0 but is not part of any $\ell^p$ space. To see this, note that $\log (i+1) $ can be bounded above by $i^{1/p}$ and hence, 
    \[\frac{1}{\log(i+1)} \geq i^{-1/p}.\]
    Thus, we have
    \[\sum_{i} \left(\frac{1}{\log(i+1)} \right)^p \geq \sum_i \left( \frac{1}{i^{1/p}}\right)^p \rightarrow \infty\]
\end{proof}

\begin{question}
    Find a sequence which is in $\ell^p$ with $p \g 1$, but $x \notin \ell^1$. 
    \label{section1.2-5}
\end{question}
\begin{proof}
    We wish to find a sequence $x = (\xi_i)$ such that $\sum \abs{xi_i}^p$ converges for all $p \g 1$ but does not converge for $p = 1$. The simplest example is $\xi_i = i^{-1}$, the sum of which diverges.
\end{proof}

\begin{question}
    The diameter $\delta(A)$ of a nonempty set $A$ in a metric space $(X,d)$ is defined to be 
    \[\delta(A) = \sup_{x , y \in A} d(x,y)\]
    $A$ is said to be bounded if $\delta(A) \l \infty$. Show that $A \subset B$ implies $\delta(A) \leq \delta(B)$.
    \label{section1.2-6}
\end{question}
\begin{proof}
    First, by the definition of supremum, we have
    \[\sup X = s \implies \forall x \in X \;,\; x \leq s\]
    Thus, over the set $B$, if $\delta(B)$ is the diameter of $B$, we have that
    \[\forall (x , y) \in B \times B , d(x,y) \leq \delta(B).\]
    Since, $A \subset B$, we have that
    all points in $A$ also belong to $B$ and hence, satisfy the property above, i.e
    \[\forall (x,y) \in A \times A \;,\; d(x,y) \leq \delta(B)\]
    and hence, taking the supremum on both sides, we get
    \[\delta(A) = \sup_{x,y \in A} d(x,y) \leq \sup_{x , y \in A}\delta(B) = \delta(B).\]
\end{proof}

 \begin{question}
    Show that $\delta(A) = 0$ if and only if $A$ consists of a single point.
    \label{section1.2-7}
\end{question}
\begin{proof}
    We wish to show
    \[\delta(A) = 0 \iff \abs{A} = 1.\]
    We first show RHS implies LHS. Suppose $\delta(A) = 0$, which means
    \[\sup_{x,y \in A} d(x,y) = 0 \implies d(x,y) \leq 0 \; \forall x,y \in A.\]
    However, since $d(x,y)$ is always non-negative, we have that 
    \[d(x,y) = 0 \; \forall x,y \in A \implies x = y \; \forall x,y \in A.\]
    This tells us that there is only one unique point in $A$.

    We now show LHS implies RHS. Suppose $A$ has only one point $x$, then, 
    \[\delta(A) =\sup_{A} d(x,y) = \sup \{d(x,x)\} = \sup \{0\} = 0\]
    This finishes the proof.
\end{proof}

\begin{question}
    The distance $D(A,B)$ between two non-empty sets of a metric space $(X,d)$ is defined to be 
    \[D(A,B) = \inf_{\substack{a \in A \\ b \in B}} d(a,b)\]
    Show that D does not define a metric on the power set of $X$.
    \label{section1.2-8}
\end{question}
\begin{proof}
    An easy way to see that this does not define a metric on the power set of $X$ (denoted by $\mathcal{P}(X)$) is as follows: clearly, the metric is non-negative and symmetric. However, let $A , B \in \mathcal{P}(X)$. If $A \cap B \neq \emptyset$, then $\exists x \in X$ such that $x \in A$ and $x \in B$. In such a case,
    \[D(A,B) = \inf_{\substack{a \in A \\ b \in B}} d(a,b) = d(x,x) = 0.\]
    However, $A \cap B \neq \emptyset \nRightarrow A = B$, and hence, the condition
    \[D(A,B) = 0 \iff A = B\]
    fails.
\end{proof}

\begin{question}
    If $A \cap B \neq \emptyset$, show that $D(A,B) = 0$. What about the converse?
    \label{section1.2-9}
\end{question}
\begin{proof}
    Let $x \in A$ and $x \in B$. Then, 
    \[\{d(a,b)\}_{\substack{a\in A \\ b \in B}} = \{d(x,x)\} \cup \{d(a,b)\}_{(a,b) \in A \times B \setminus 
    \{(x,x)\}} = \{0\} \cup \{d(a,b)\}_{(a,b) \in A \times B \setminus 
    \{(x,x)\}} \]
    and hence, 
    \[\inf_{\substack{a \in A \\ b \in B}} d(a,b) = 0 = D(A,B).\]
    The converse may not be true. The idea comes from accumulation points (defined in the next section). $x$ is said to be an accumulation point of a set $A$ if every neighborhood of $x$ contains atleast one point from $A$ distinct from $x$. Let $x \in B$ and $x \notin A$ be an accumulation point of $A$. Then, every $\epsilon-$neighborhood of $x$ will contain $a(\epsilon) \in A$ and hence, for every $\epsilon \g 0$, you can find $a(\epsilon)$ such that $d(a(\epsilon) , x) = \epsilon$. Since $\epsilon$ can be arbitrarily small, the infimum becomes zero. However, $A \cap B = \emptyset$. Thus, the converse may not be true. 
\end{proof}

\begin{question}
    The distance $D(x)$ from a point $x$ to a non-empty subset $B$ of $(X,d)$ is defined as 
    \[D(x,B) = \inf_{b \in B} d(x,b)\]
    Show that for any $x , y \in X$, 
    \[\abs{D(x,B) - D(y,B)} \leq d(x,y).\]
    \label{section1.2-10}
\end{question}
\begin{proof}
    First, consider the case when $x \in B$ and $y \in B$. Clearly, $D(x,B) = D(y,B) = 0$ and hence, 
    \[\abs{D(x,B) - D(y,B)} = 0 \leq d(x,y).\]
    by the properties of $d$.
    Now, consider the case when $x \notin B$ and $y \in B$. Clearly, $D(y,b) = 0$ and hence, 
    \[\abs{D(x,B)} = \inf_{b \in B} d(x,b) \leq d(x,y) \]
    since $y \in B$.
    Finally, consider the case when $x \notin B$ and $y \notin B$. Then, we have
    \[D(x,B) = \inf_{b \in B} d(x,b) \leq \inf_{b \in B}\left( d(x,y) + d(y,b)\right) = d(x,y) + \inf_{b \in B} d(y,b) = d(x,y) + D(y,B).\]
    where the first inequality follows from the triangle inequality applied on $d$. Rearranging the terms gives us
    \[D(x,B) - D(y,B) \leq d(x,y).\]
    By symmetry, interchanging $x$ and $y$ results in 
    \[D(y,B) - D(x,B) \leq d(x,y).\]
    This finishes the proof.
\end{proof}

\begin{question}
    If $(X,d)$ is any metric space, show that another metric on $X$ is defined by 
    \[\tilde{d}(x,y) = \frac{d(x,y)}{1 + d(x,y)}\]
    and $X$ is bounded in the metric $\tilde{d}$.
    \label{section1.2-11}
\end{question}
\begin{proof}
    First, the metric $\tilde{d}$ is non-negative, symmetric and satisfies $\tilde{d}(x,y) = 0 \iff x = y$ from the properties of $d$. Also
    \[\lim\limits_{d(x,y) \rightarrow \infty} \tilde{d}(x,y) = 1\] 
    and hence, it does remain bounded. Also, notice that
    \[f(t) = \frac{t}{1 + t}\]
    is always increasing and hence, by the non-negativity of the metric $d$, we have
    \begin{align*}
        f(d(x,y)) \leq f(d(x,z) + d(z,y)) = \frac{d(x,z) + d(z,y)}{1 + d(x,z) + d(z,y)}  &= \frac{d(x,z)}{1 + d(x,z) + d(z,y)} + \frac{d(z,y)}{1 + d(x,z) + d(z,y)}
        \\
        &\leq \frac{d(x,z)}{1 + d(x,z)} + \frac{d(z,y)}{1 + d(x,z)}
    \end{align*}
    which finishes the proof.
\end{proof}


\begin{question}
    Show that the union of two bounded sets is also bounded.
    \label{section1.2-12}
\end{question}
\begin{proof}
    Let $A$ and $B$ be two bounded sets such that $\delta(A) \l \infty$ and $\delta(B) \l \infty$. Then, using the definition of $\delta(A)$ from \ref{section1.2-6}, we have
    \[\forall a_1 , a_2 \in A, d(a_1,a_2) \leq \delta(A) \l \infty \text{ and } \forall b_1 , b_2 \in B, d(b_1,b_2) \leq \delta(B) \l \infty\]
    Thus, for any $a_1 \in A , b_1 \in B$, we have
    \[d(a_1 , b_1) \leq d(a_1 , a_2) + d(a_2,b_2) + d(b_1 , b_2) \leq \delta(A) + \delta(B) + d(a_2 , b_2)\]
    and since $d$ is a metric, it is also bounded, resulting in $d(a_1,b_1)$ being bounded.
\end{proof}

\begin{question}
    The cartesian product $X = X_1 \times X_2$ of two metric spaces $(X_1 , d_1)$ and $(X_2,d_2)$ can be made into a metric space $(X,d)$ in many ways. For example, show that a metric $d$ is defined by 
    \[d(x,y) = d_1(x_1,y_1) + d_2(x_2,y_2)\]
    where $x = (x_1 , x_2)$ and $y = (y_1 , y_2)$.
    \label{section1.2-13}
\end{question}
\begin{proof}
    First, note that $d$ is bounded, non-negative, and symmetric owing to the metric properties of $d_1$ and $d_2$. Also, due to the metric properties of $d_1$ and $d_2$, we have
    \[d(x,y) = 0 \iff d_1(x_1,y_1) = 0 \text{ and } d_2(x_2 , y_2) = 0 \iff x = (x_1,x_2) = (y_1,y_2) = y.\]
    Finally, we show the triangle inequality for some $z = (z_1 , z_2)$ as follows:
    \[d(x,y) = d_1(x_1,y_1) + d_2(x_2,y_2) \leq d_1(x_1 , z_1) + d_1(z_1,y_1) + d_2(x_2,z_2) + d_2(z_2 , y_2) = d(x,z) + d(z,y).\]
    This finishes the proof.
\end{proof}

\begin{question}
    Show that another metric on $X$ is defined by
    \[\tilde{d}(x,y) = \sqrt{d_1(x_1,y_1)^2 + d_2(x_2,y_2)^2}\]
    \label{section1.2-14}
\end{question}
\begin{proof}
    Once again, note that $\tilde{d}$ is bounded, non-negative, symmetric and satisfies $\tilde{d}(x,y) = 0 \iff x = y$ owing to the metric properties of $d_1$ and $d_2$.
    Finally, we show the triangle inequality for some $z = (z_1 , z_2)$ as follows:
    \begin{align*}
        \tilde{d}(x,y) &= \sqrt{d_1(x_1,y_1)^2 + d_2(x_2 , y_2)^2}
        \\
        &\leq \sqrt{(d_1(x_1,z_1) + d_1(z_1,y_1))^2 + (d_2(x_2,z_2) + d_2(z_2,y_2))^2} 
        \\
        &\leq \sqrt{d_1(x_1,z_1)^2 + d_2(x_2,z_2)^2 + d_1(z_1,y_1)^2 + d_2(z_2,y_2)^2 + 2d_1(x_1,z_1)d_1(z_1,y_1) + 2d_2(x_2,z_2)d_2(z_2,y_2)}
    \end{align*}
    Using Hölder's inequality, which says that for $a_i , b_i \geq 0$ and $p,q$ such that $p^{-1} + q^{-1} = 1$, 
    \[\sum_{i=1}^N a_i b _i \leq \left( \sum_{i=1}^N a_i^p \right)^{\frac{1}{p}} + \left( \sum_{i=1}^N b_i^q \right)^{\frac{1}{q}}\]
    we get
    \[d_1(x_1,z_1)d_1(z_1,y_1) + d_2(x_2,z_2)d_2(z_2,y_2) \leq \sqrt{d_1(x_1,z_1)^2 + d_2(x_2,z_2)^2} \sqrt{d_1(z_1,y_1)^2 + d_2(z_2,y_2)^2}\]
    and hence, we get
    \[\tilde{d}(x,y) \leq \sqrt{d_1(x_1,z_1)^2 + d_2(x_2,z_2)^2} \sqrt{d_1(z_1,y_1)^2 + d_2(z_2,y_2)^2} = \tilde{d}(x,z) + \tilde{d}(z,y).\]
    This finishes the proof.
\end{proof}

\begin{question}
    Show that another metric on $X$ is defined by
    \[\tilde{\tilde{d}}(x,y) = \max\{d_1(x_1,y_1) , d_2(x_2,y_2)\}\]
    \label{section1.2-15}
\end{question}
\begin{proof}
    Once again, note that $\tilde{\tilde{d}}$ is bounded, non-negative, symmetric and satisfies $\tilde{\tilde{d}}(x,y) = 0 \iff x = y$ owing to the metric properties of $d_1$ and $d_2$.
    Finally, we show the triangle inequality for some $z = (z_1 , z_2)$ as follows:
    \begin{align*}
        \tilde{\tilde{d}}(x,y) &= \max\{d_1(x_1,y_1) , d_2(x_2,y_2)\}
        \\
        &\leq \max\{d_1(x_1,z_1) + d_1(z_1,y_1) , d_2(x_2,z_2) + d_2(z_2,y_2)\}
        \\
        &= d_i(x_i,z_i) + d_i(z_i,y_i) \; i \in \{1,2\}
        \\
        &\leq \max\{d_1(x_1,z_1), d_2(x_2,z_2)\} + \max\{ d_1(z_1,y_1) , d_2(z_2,y_2)\}
        \\
        &= \tilde{\tilde{d}}(x,z) + \tilde{\tilde{d}}(z,y)
    \end{align*}
    This finishes the proof.
\end{proof}


