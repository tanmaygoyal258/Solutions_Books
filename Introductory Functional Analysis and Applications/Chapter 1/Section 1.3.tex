\subsection{Open set, Closed set, Neighborhood}

\begin{question}
    Justify the terms \emph{open ball} and \emph{closed ball} by showing that an open ball is an open set and a closed ball is a closed set.
    \label{section1.3-1}
\end{question}
\begin{proof}
    Let $B(x_0 , r)$ be a ball of radius $r$ around $x_0$. If we wish to show that $B(x_0 , r)$ is open, it suffices to show that we can draw a ball around any point of $B(x_0,r)$. In other words, choose any point $y_0 \in B(x_0 , r)$. For some radius $r^\prime, B(y_0 , r^\prime) \subseteq B(x_0 , r)$.
    Let $r^\prime \leq r - d(x_0 , y_0)$. Then, for some $x \in B(y_0 , r^\prime)$, we have
    \[d(x,x_0) \leq d(x,y_0) + d(y_0,x_0) \leq r\]
    and hence, $x \in B(x_0 , r)$. This shows that an open ball is also an open set.

    Showing that a closed ball is a closed set is equivalent to showing the complement of a closed ball is open. Let $X$ be a set and $x_0 \in X$. Let $\tilde{B}(x_0 , r)$ denote a closed ball of radius $r$ in $X$ and let $\tilde{B}(x_0 , r)^C$ denote the complement of $\tilde{B}(x_0 , r)$. 
    Choose some point $y_0 \in \tilde{B}(x_0 , r)^C$ and $r^\prime \leq d(x_0 , y_0) - r$. Clearly, $d(x_0 , y_0) \geq r$. Then, $B(y_0,r^\prime)$ is a ball about $y_0$ with radius $r^\prime$. Let $x \in B(y_0,r^\prime$. Then,
    \[d(x_0,y_0) \leq d(x_0,x) + d(y_0,x) \implies d(x_0,x) \geq d(x_0,y_0) - d(y_0,x) \geq r - d(y_0,x) \geq r.\]
    and hence, $x \in \tilde{B}(x_0,r)^C$. Thus, the complement of the closed ball is open. This finishes the proof.
\end{proof}

\begin{question}
    What is an open ball $B(x_0  ,1)$ in $\R$ and $\C$? What is an open ball in $C[a,b]$?
    \label{section1.3-2}
\end{question}
\begin{proof}
    In $\R$, the open ball is the set of points $B(x_0 , 1) = \{x_0 \pm \epsilon \mid \abs{\epsilon} \l 1\}$. In $\C$, let $x_0 = a + \iota b$, then the open ball is defined as $B(x_0 , 1) = \{(a + \epsilon_1) + \iota (b + \epsilon_2) \mid \sqrt{\epsilon_1^2 + \epsilon_2^2} \l 1 \}$. Finally, let $x_0 \in C[a,b]$ be a function of $t$. Then, under the metric
    \[d(x,y) = \max_{t \in [a,b]} \abs{x(t) - y(t)}\]
    we have the open ball to be defined as
    \[B(x_0 , 1) = \left\{x_0(t) \pm \epsilon(t) \mid \abs{\max_{t \in [a,b]}\epsilon(t)} \l 1\right\}.\]
\end{proof}

\begin{question}
    Consider $C[0,2\pi]$ and determine the smallest $r$ such that $y \in \tilde{B}(x,r)$ where $x(t) = \sin t$ and $y(t) = \cos t$.
    \label{section1.3-3}
\end{question}
\begin{proof}
    We wish to find the smallest value of $r$ such that $y(t) \in \tilde{B}(x(t) , r)$. Let the metric on $C[0,2\pi]$ be defined as
    \[d(x,y) = \max_{t \in [0,2\pi]} \abs{x(t) - y(t)}.\]
    Then, we wish to find the smallest value of $r$ such that $d(x,y) \leq r$. Substituting the values of $x(t)$ and $y(t)$, we get
    \[\max_{t \in [0,2\pi]} \abs{\sin t - \cos t} \leq r\]
    Differentiating $\sin t - \cos t$ w.r.t $t$ and setting to $0$, we find that the maximum value of the expression is obtained at $t = 3\pi/4$ and the maximum value is $\sqrt{2}$. Thus, setting $r = \sqrt{2}$ ensures $y \in B(x , r)$.
\end{proof}

\begin{question}
    Show that any nonempty set $A \subset (X,d)$ is open if and only if it is a union of open balls.
\label{section1.3-4}
\end{question}
\begin{proof}
    We first assume that a nonempty set $A \subset (X,d)$ is open and wish to show it is the union of open balls. Choose any point $x \in A$. Then, there exists a ball $B$ around $x$ such that $B \subset A$. Since $x$ is arbitrary, at each point in $A$, there exists a ball comprised entirely in $A$ and hence, $A$ is the union of all such open balls. 
    
    Now we assume $A$ is a union of open balls. For some open ball $B$, $\exists x \in B$ such that we can draw another ball $B_0$ around $x$ and $B_0 \subset B$. Since $A$ is the union of open balls, $x \in A$ and $B_0 \subset B \subset A$. Since $x$ is arbitrary, $A$ is an open set.
\end{proof}

\begin{question}
    It is important to realize that some sets may be open and closed at the same time. Show that this is always the case for $X$ and $\emptyset.$ Show that in a discrete metric space $X$, every subset is open and closed.
    \label{section1.3-5}
\end{question}
\begin{proof}
    Note that $\emptyset$ is always open since there are no points in  it. This also ensures that the set $X$ is closed (since it's complement $\emptyset$ is open). On the other hand, $X$ is open because it consists of the entire space and all limit points, and hence, it's complement $\emptyset$ is closed.

    In a discrete metric space, let $A \subset X$. $A$ is open if we can draw a ball around each point in $A$. Let $r \l 1$. Then, for each $a \in A$, $B(a,r) = \{a\} \subset A$, and hence, the subsets are open. In a similar fashion, let $A^C$ be the complement of $A$, and let $a \in A^C$ be some point. Then, choosing $r \l 1$, we have a ball $B(a,r) = \{a\} \subset A^C$, and hence, $A^C$ is also open, resulting in $A$ being closed.
\end{proof} 

\begin{question}
    If $x_0$ is an accumulation point of $A \subset (X,d)$, show that any neighborhood of $x_0$ contains infinitely many points of $A$.
    \label{section1.3-6}
\end{question}
\begin{proof}
     We shall prove the claim by contradiction. By the definition of an accumulating point, the neighborhood of $x_0$ contains at least one point distinct from $x_0$ from $A$. Let $B(x_0 , \epsilon)$ be an $\epsilon-$neighborhood of $x_0$, which consists of finitely many points ($\neq x_0$)  $\{a_1 , \ldots ,a_k\}$. Let $r \leq \min_{j \in [k]} d(x_0 , a_j)$. Then, the $r-$neighborhood of $x_0$ consists of no points other than $x_0$, which is a contradiction since $x_0$ is an accumulation point. Hence, there should be another point in the $r-$neighborhood, which contradicts our assumption of finitely many points.
\end{proof}

\begin{question}
    Describe the closure of each of the sets: the integers on $\R$, the rational numbers on $\R$, the complex numbers with rational real and imaginary parts in $\C$, and the disk $\{z \mid \abs{z} \l 1\} \subset \C$.
\label{section1.3-7}
\end{question}
\begin{proof}
    The closure of a set includes all points of the set and all it's limit points. The limit points of a set are points such \emph{every} neighborhood of the point consists of at least one element of the set distinct from the point. Clearly, for the set of integers on $\R$, the closure is the set of integers itself. In other words, there are no limit points for the set of integers. Let $z$ be an integer, then drawing an $\epsilon-$neighborhood around $z$ with $\epsilon \l 1$ ensures that there is no integer apart from $z$ in this neighborhood. Same is the case for any rational number $q$, any $\epsilon-$neighborhood around $q$ may not consist of an integer.

    For the set of rationals on $\R$, the closure is $\R$. This is because choosing any point in $\R$ will have atleast one rational number in it's neighborhood. Same is the case for the set of complex numbers with rational real and imaginary parts. You can always find a new complex number with rational real and imaginary parts in a neighborhood. 

    Finally, for the open disk of radius $1$ in the complex numbers, defined as
    \[\{z \mid \abs{z} \l 1\} \subset \C\]
    the accumulation points will be 
     \[\{z \mid \abs{z} \leq 1\}.\]
     Since any neighborhood of a "boundary point" will intersect with the disk, they also are accumulation points, and hence, the closure is simply the closed disk of radius 1.
\end{proof}

\begin{question}
    Show that the closure $\overline{B(x_0 , r)}$ of an open ball $B(x_0 , r)$ in a metric space can differ from the closed ball $\tilde{B}(x_0 , r)$.
    \label{section1.3-8}
\end{question}
\begin{proof}
    \textcolor{red}{TODO}.
\end{proof}

\begin{question}
    Show that $A \subset \overline{A}$ , $\overline{A \cup B} = \overline{A} \cup \overline{B}$, and $\overline{A \cap B} \subset \overline{A} \cap \overline{B}$.
    \label{section1.3-9}
\end{question}
\begin{proof}
    $A \subset \overline{A}$ follows from the definition of a closure. Now, we show $\overline{A \cup B} = \overline{A} \cup \overline{B}.$ To show this, we show that $\overline{A \cup B} \subset \overline{A} \cup \overline{B}$ and $\overline{A} \cup \overline{B} \subset \overline{A \cup B}$. First, let $x \in \overline{A \cup B}$. Then, 
    \begin{enumerate}
        \item $x \in A \cup B$. This means that $x \in A$ or $x \in B$. WLOG, assume $x \in A$, then by the definition of closure, $x \in \overline{A}$, and hence, $x \in \overline{A} \cup \overline{B}$.
        \item $x \notin A \cup B$ and $x$ is an accumulation point of $A \cup B$. This means that $x$ is also an accumulation point of either $A$ or $B$. This is clear if every neighborhood of $x$ contains some point of $A$ or if every neighborhood contains a point of $B$. Now consider the case where some $\epsilon_A-$neighborhood of $x$ contains only points of $A$ and $\epsilon_B-$neighborhood contains points of only $B$. WLOG, let $\epsilon_A \geq \epsilon_B$. Then, $\epsilon_A-$neighborhood also contains points from $B$. Thus, $x$ would be a accumulation point for either $A$ or $B$, which implies that $x \in \overline{A}$ or $x \in \overline{B}$, resulting in $x \in \overline{A} \cup \overline{B}$. 
    \end{enumerate}
    This shows that $x \in \overline{A} \cup \overline{B}$. 

    Now we show that $\overline{A} \cup \overline{B} \subset \overline{A \cup B}$. 
    Let $x \in \overline{A} \cup \overline{B}$. WLOG, assume $x \in \overline{A}$. Then, 
    \begin{enumerate}
        \item $x \in A$, which means that $x \in A \cup B$ and hence, $x \in \overline{A \cup B}$. 
        \item $x \notin A$ and $x$ is an accumulation point for $A$. This means that every neighborhood of $x$ contains some point from $A$. That means that every neighborhood of $x$ also contains some point from $A \cup B$, and hence, $x$ is an accumulation point for $A \cup B$. Thus, $x \in  \overline{A \cup B}$.
    \end{enumerate}
    Thus, $\overline{A} \cup \overline{B} \subset \overline{A \cup B}$. This finishes the proof. 

    Finally, we show that $\overline{A \cap B} \subset \overline{A} \cap \overline{B} $. Let $x \in \overline{A \cap B}$. Then, 
    \begin{enumerate}
        \item $x \in A \cap B$. This means that $x \in A$ and $x \in B$, and hence, $x \in \overline{A}$ and $x \in \overline{B}$, resulting in $x \in \overline{A} \cap \overline{B}$.
        \item $x \notin A \cap B$ but $x$ is an accumulating point of $A \cap B$. That means that every neighborhood of $x$ has a point from $A \cap B$, which means the point in the neighborhood belongs to both $A$ and $B$. Thus, every neighborhood contains a point from $A$ and $B$, making $x$ an accumulation point for both $A$ and $B$. Hence, $x \in \overline{A}$ and $x \in \overline{B}$, resulting in $x \in \overline{A} \cap \overline{B}$.
    \end{enumerate}
    This finishes the proof.
\end{proof}

\begin{question}
    A point $x$ not belonging to a closed set $M \subset (X,d)$ always has a nonzero distance from $M$. To prove this, show that $x \in \overline{A}$ if and only if $D(x,A) = 0$, where $A$ is any nonempty subset.
\label{section1.3-10}
\end{question}
\begin{proof}
    First, we define 
    \[D(x,A) = \inf_{a \in A}d(x,a).\]
    We show that $x \in \overline{A} \implies D(x,A) = 0$. If $x \in A$, then by definition of $D$, we have $D(x,A) = 0$. However, let $x \notin A$, then for every $\epsilon-$neighborhood of $x$, $B(x,\epsilon) \cap A \neq \emptyset$, which means that for any $\epsilon \g 0$, we can find a point distinct from $x$ belonging to $A$. As $\epsilon$ can be arbitrarily small, the infimum would be $0$.

    Next we show that $D(x,A) = 0 \implies x \in \overline{A}$. If $x \in A$, then it is trivial. However, if $x \notin A$, then we have that for all $\epsilon \g 0$, there exists some point $a(\epsilon) \in A$ such that $d(a(\epsilon),x) = \epsilon$. Since $\epsilon$ can be made arbitrarily small, the infimum results in $0$. However, this also means that for every $\epsilon-$neighborhood of $x$, there exists some point $a(\epsilon) \in A$ and hence, $x$ is an accumulation point of $A$. Thus, $x \in \overline{A}$.
    This finishes the proof.
\end{proof}

\begin{question}
    A boundary point of $A \subset (X,d)$ is a point of $X$ (which may or may not be in $A$) such that every neighborhood of $x$ contains points of $A$ as well as points not belonging to $A$. The boundary of $A$ is defined to be the set of all boundary points of $A$. Find the boundaries of $(-1,1)$, $[-1,1)$ , $[-1,1]$ on $\R$, the set of rationals on $\R$, and the disks $\{z \mid \abs{z} \l 1\}$ and $\{z \mid \abs{z} \leq 1\} \subset \C$.
\label{section1.3-11}
\end{question}
\begin{proof}
    The boundary points of $(-1,1)$ are $\{-1,1\}.$ This is because any $\epsilon-$neighborhood of $-1$ will contain $-1+\epsilon \in (-1,1)$ and $-1-\epsilon \notin (-1,1)$. Note that $x \in (-1,1)$ is not a boundary point because every neighborhood will not contain any point which does not belong to $(-1,1)$. The boundary points for $(-1,1]$ and $[-1,1]$ are also $\{-1,1\}$.

    The boundary points for the set of rationals on $\R$ is $\R$. This is because for any $q \in \R$, the neighborhood of $q$ will contain rationals as well as irrationals. 

    Finally, the boundary points for the disks $\{z \mid \abs{z} \l 1\}$ and $\{z \mid \abs{z} \leq 1\} \subset \C$ is simply the set $\{z \mid \abs{z} = 1\}$.
\end{proof}

\begin{question}
    Show that $B[a,b]$ is not separable.
    \label{section1.3-12}
\end{question}
\begin{proof}
    We shall use the same methodology as the one used to show $\ell^p$ is not separable. We will first construct an uncountable set of functions $x_n$ such that $d(x_n , x_m) \g \epsilon.$ Now, if we take balls of radius $\frac{\epsilon}{3}$ around each $x_i$, we will have uncountable balls that do not overlap. Suppose $M$ is a dense subset, then by the definition of dense, each of the balls around $x_i$ should have an element of $M$. This means that $M$ has to intersect with every ball and since the number of balls are uncountable and disjoint, $M$ is uncountable. Since any arbitrary dense set is uncountable, the space is not separable.

    Now, to construct this uncountable set of functions, note that $[a,b]$ is an uncountable set, and hence, having functions labeled with $c \in [a,b]$ would result in uncountable functions. Define 
    \[x_c(d) = \mathbbm{1}\{c=d\} \; \forall d \in [a,b].\]
    This function is bounded, and moreover there are uncountable number of functions. Further, 
    \[d(x_m , x_n) = \sup_{t \in [a,b]} \abs{x_m(t) - 
    x_n(t)} = 1\]
    and hence, substituting $\epsilon = 1$ will finish the proof.
\end{proof} 

\begin{question}
    Show that a metric space $X$ is separable if and only if $X$ has a countable subset $Y$ with the following property: For every $\epsilon \g 0$ and every $x \in X$, there is a $y \in Y$ such that $d(x,y) \l \epsilon$.
    \label{section1.3-13}
\end{question}
\begin{proof}
    We first show that if a metric space $X$ is separable, then it contains a countable subset with the property mentioned above. First, if $X$ is separable it would contain separable dense subsets $Y$. Since $Y$ is dense, we have that $\overline{Y} = X$, which means that every point in $x \in X$ either belongs to $Y$ or is an accumulation point of $Y$. In such a case, if $x \in Y$, then it is trivial. However, if $x \notin Y$ and is an accumulation point, then by the definition of an accumulation point, every $\epsilon-$neighborhood of $x$ would contain some $y \in Y$, and hence, $d(x,y) \l \epsilon$. This finishes the proof.

    We now prove the converse, i.e if there is a countable subset $Y$ that satisfies the above mentioned property, then $X$ is separable. In such a case, the property mentions that for every $\epsilon \g 0$ and for every $x \in X$, there exists $y \in Y$ such that $d(x,y) \l 0$. If $x \in Y$, then this is trivially true. However, if $x \notin Y$, then, by the above mentioned property, $x$ is an accumulation point for $Y$, which means that all points in $X$ are either in $Y$ or are accumulation points for $Y$. Thus, $\overline{Y} = X$ and hence, $Y$ is dense in X. Since $Y$ is countable, we have that $X$ is a separable set. This finishes the proof.
    \end{proof}

\begin{question}
    Show that a mapping $T : X \mapsto Y$ is continuous if and only if the inverse image of any closed set $M \subset Y$ is a closed set in $X$.
    \label{section1.3-14}
\end{question}
\begin{proof}
    We instead show that if $T$ is a continuous mapping if and only if the inverse image of any open set in $Y$ is open in $X$.

    Suppose $T$ is a continuous mapping. Then for every $\epsilon \g 0, \exists \delta \g 0$ such that 
    \[d(x_0,x) \leq \delta \implies d(Tx_0 , Tx) \leq \epsilon.\]
    Thus, consider the point $Tx_0 \in Y$ and the set $B(Tx_0 , \epsilon)$ which is open. Clearly, due to the definition of $T$, $B(Tx_0,\epsilon)$ maps back to $B(x_0 , \delta)$ which is again open, and hence, if $T$ is continuous, then the inverse image of an open set is an open set.

    Now, we show the converse. Suppose we take the set $B(Tx_0 , \epsilon) \subset Y$, which is open. Then, we have that the inverse image of $B(Tx_0 , \epsilon)$ is also open, say some set $N \in X$ such that $x_0 \in N$. Also, since $N$ is an open set,$B(x_0 , \delta) \subset N$, and hence, the definition of continuity is fulfilled.

    Now, we shall show the statement. Suppose $T$ is continuous and $M \subset Y$ is closed. This means that $M^C$ is open and hence, the inverse image of $M$, say $A \subset X$ is open. Thus, $A^C$ is closed. Similarly, let $M \subset Y$ is closed and it's inverse image $A \subset X$ is also closed. This means that $M^C$ and $A^C$ is open and hence, $T$ is continuous.
\end{proof}

\begin{question}
Show that the image of an open set under a continuous mapping need not be open.
\label{section1.3-15}
\end{question}
\begin{proof}
    Consider a function $f$ on the set $(a,b)$ that maps to $[c,d]$ such that for some $a \l a_1,b_1 \l b$, we have $f(a_1) = c$ and $f(b_1) = d$. Any such function  will satisfy the above claim.
\end{proof}