
\subsection{Metric Space}


\begin{question}
    Show that the real line is a metric space.
\label{section1.1-1}
\end{question}
\begin{proof}
    To show that the real line is a metric space, we wish to show that there is a valid metric on this space. Obviously, the choice of the metric is $d(x,y) = \abs{x-y}$. We now wish to show the following properties:
    \begin{enumerate}
        \item $d$ is non-negative.
        \item $d(x,y) = 0 \iff x = y$.
        \item $d(x,y) = d(y,x)$.
        \item $d(x,y) \leq d(x,z) + d(z,y)$
    \end{enumerate}
    Properties (1) , (2), and (3) are obvious. We now show property (4) for the choice of $d(x,y) = \abs{x-y}.$ Consider
    \begin{align*}
        (x-y)^2 &= x^2 - 2xy + y^2
        \\
        &= x^2 - 2xz + z^2 + z^2 - 2zy + y^2 + (2xz + 2zy - 2xy - 2z^2)
        \\
        &= (x-z)^2 + (z-y)^2 + 2x(z-y) + 2z(y-z)
        \\
        &= \abs{x-z}^2 + \abs{z-y}^2  + 2(x-z)(z-y)
        \\
        &\leq \left( \abs{x-z} + \abs{y-z} \right)^2
    \end{align*}
    where the inequality follows since $x-y \leq \abs{x-y}$. Taking a square root on both sides finishes the proof.
\end{proof}

\begin{question}
    Does $d(x,y) = (x-y)^2$ define a metric on the set of all real numbers?
\label{section1.1-2}
\end{question}{
\begin{proof}
    To show that $d(x,y)$ is a valid metric, we must ensure that it is non-negative, symmetric, satisfies the triangle inequality, and satisfies the condition that $(x-y)^2 = 0 \iff x = y$. We see that the first, second, and last conditions are always true. We only need to check for the triangle inequality. Notice, from the answer of \ref{section1.1-1}, we have
    \[(x-y)^2 = (x-z)^2 + (y-z)^2 + 2(x-z)(z-y)\]
    WLOG, assume $x \leq y$. If $z \in [x,y]$, then we have $(x-z) \leq 0$ and $(z-y) \leq 0$, and hence, $(x-z)(z-y) \geq 0$, which results in 
    \[d(x,y) \geq d(x,z) + d(z,y)\]
    which is a contradiction of the triangle inequality. Hence, $d(x,y) = (x-y)^2$ is not a metric.
\end{proof}

\begin{question}
    Show that $d(x,y) = \sqrt{\abs{x-y}}$ defines a metric on the set of all real numbers.
\label{section1.1-3}
\end{question}
\begin{proof}
    Once again, we can check that $d(x,y)$ satisfies non-negativity, symmetry, and the property that $d(x,y) = 0 \iff x = y$. We wish to show $d$ satisfies the triangle inequality. Since $\abs{x-y}$ is a valid metric (refer \ref{section1.1-1}), using the triangle inequality for this along with the fact that $\sqrt{.}$ is monotonically increasing, we get
    \[\sqrt{\abs{x-y}} \leq \sqrt{\abs{x-z} + \abs{z-y}} \leq \sqrt{\abs{x-z}} + \sqrt{\abs{z-y}}\]
    where the last inequality follows from the fact that if $a , b \geq 0$, then 
    \[\sqrt{a + b} \leq \sqrt{a + 2\sqrt{ab} + b} = \sqrt{(\sqrt{a} + \sqrt{b})^2} = \sqrt{a} + \sqrt{b}\].
\end{proof}

\begin{question}
    Find all metrics on a set $X$ consisting of two points and only one point.
    \label{section1.1-4}
\end{question}
\begin{proof}
    If $X$ consists of only one point, then the only metric that can be defined keeping in mind all properties (note that the triangle inequality does not apply since we do not have a second point for this) is $d(x,x) = 0$.

    If $X$ consists of two points, once again, the triangle inequality does not apply, and hence, it is easy to check that 
    \[d(x,y) = 
    \begin{cases}
        0 & x = y
        \\
        c & x \neq y
    \end{cases}.\]
\end{proof}

\begin{question}
    Let $d$ be a metric on $X$. Determine all constants $k$ such that $kd$ and $d + k$ is a metric on $X$.
    \label{section1.1-5}
\end{question}
\begin{proof}
    Note that for $kd$ to remain a metric, all we need is $k \geq 0$ to ensure it is non-negative. All other properties continue to hold as it is.

    For $d + k$ to be a metric, the simplest starting point is that $(d+k)(x,y) = 0 \iff x = y$. However, we know that $d(x,y) = 0 \iff x = y$, and since, $k$ is a constant, this property is satisfied only if $k(x,y) = 0$. Hence, no constant $k$ can be added such that $d + k$ is also a metric. 
\end{proof}

\begin{question}
    Show that $d$ in 1.1-6 satisfies the triangle inequality.
    \label{section1.1-6}
\end{question}
\begin{proof}
    The metric in 1.1-6 is for a sequence space $\ell^\infty$ such that for $x = (\xi_i)$ and $y = (\eta_i)$ such that $\xi \leq c_x$ and $\eta_i \leq c_y$, we have
    \[d(x,y) = \sup\limits_{i \in \N} \abs{\xi_i -\eta_i}\]
    Note that this is always non-negative and symmetric, owing to the absolute difference of the terms (since absolute difference is always non-negative, the least upper bound is also non-negative). Also, if $x = y$,. i.e, $\xi_i = \eta_i$ for all $i$, then the supremum is $0$. Finally, we show the triangle inequality as follows. For some $z = (\zeta_i)$, we have
    \[d(x,y) = \sup_{i \in \N} \abs{\xi_i - \eta_i} \leq \sup_{i \in \N} \left( \abs{\xi_i - \zeta_i} + \abs{\zeta_i - \eta_i} \right) \leq \sup_{i \in \N} \abs{\xi_i - \zeta_i} + \sup_{i \in \N} \abs{\zeta_i - \eta_i} = d(x,z) + d(z,y)\]
    where we use the triangle inequality for $\abs{x-y}$ (check \ref{section1.1-1}) and the fact that $\sup (f+g)(x) \leq \sup f(x) + \sup g(x)$.
\end{proof}

\begin{question}
    If $A$ is the subspace of $\ell^\infty$ consisting of all sequences of zeros and ones, then what is the induced metric on $A$?
    \label{section1.1-7}
\end{question}
\begin{proof}
    If $A$ is the subspace consisting of all sequences only containing zeros and ones, then the coordinate-wise absolute difference of any two sequences can only take on two values; $0$, when the coordinates are the same, and $1$ when the coordinates are different. Thus, the supremum of this absolute difference is $1$. In other words, the metric can be expressed as:
    \[d(x,y) = \mathbbm{1}\{x = y\}\]
    which is exactly the discrete metric.
\end{proof}

\begin{question}
    Show that another metric $\tilde{d}$ on the space $X$ in 1.1-7 is defined by 
    \[\tilde{d}(x,y) = \int_{a}^{b} \abs{x(t) - y(t)} \; dt\]
    \label{section1.1-8}
\end{question}
\begin{proof}
    We wish to show that the metric $d(x,y)$ is a defined on the space of continuous functions $C[a,b]$, i.e $x$ and $y$ are now continuous functions. Note that $d(x,y)$ is clearly non-negative and symmetric owing to the point-wise absolute difference of the functions. Further, if $x(c) = y(c) \; \forall c \in [a,b]$, then, $d(x,y) = \int_{a}^{b} 0 dt = 0$. Thus, we wish to show the triangle inequality now. Using the fact that the absolute difference is a metric (check \ref{section1.1-1}), using the triangle inequality, we have
    \[\int_{a}^{b} \abs{x(t) - y(t)} \; dt \leq \int_{a}^{b} \left( \abs{x(t) - z(t)} + \abs{z(t) - y(t)} \right) \; dt = \int_{a}^{b} \abs{x(t) - z(t)} \; dt + \int_{a}^{b} \abs{z(t) - y(t)} \; dt\]
    which proves that $d(x,y)$ satisfies the triangle inequality.
 \end{proof}

 \begin{question}
     Show that $d$ in 1.1-8 is a metric.
     \label{section1.1-9}
 \end{question}
 \begin{proof}
     We wish to show that the discrete metric, defined as
     \[d(x,y) = 
     \begin{cases}
         0 & x = y
         \\
         1 & x \neq y
     \end{cases}\]
     is a metric. Clearly, this is non-negative and symmetric. Further, we have that $d(x,y) = 0 \iff x = y$. Finally, we also have 
     \begin{enumerate}
         \item If $x \neq y$, then $d(x,y) = 1$. In such a case, if $x = z$ or $y = z$, then $d(x,z) + d(y,z) = 1$. However, if $x \neq z \neq y$, then $d(x,z) + d(y,z) = 2$. Thus, $d(x,y) \leq d(x,z) + d(z,y)$.
         \item If $x = y$, then $d(x,y) = 0$. Now, if $x = y \neq z$, then $d(x,z) + d(z,y) = 2$, and if $x = y = z$, then $d(x,z) + d(z,y) = 0$, and in both cases, $d(x,z) + d(z,y) \geq d(x,y)$.
     \end{enumerate}

     Thus, the triangle inequality holds for $d$, and hence, it is a valid metric.
\end{proof}

\begin{question}
    Let $X$ be the set of all ordered triples of zeros and ones. Show that $X$ consists of eight elements and a metric $d$ on $X$ is defined by $d(x,y) = $number of places where $x$ and $y$ have different entries. 
    \label{section1.1-10}
\end{question}

\begin{proof}
    Clearly, $X$ has $2^3 = 8$ elements. We wish to show that 
    \[d(x,y) = \sum_{i=1}^3 \mathbbm{1}\{x_i \neq y_i\}\]
    is a metric on $X$ where $x_i$ represents the $i^{th}$ coordinate of $x$. Clearly, this metric is non-negative, symmetric, and satisfies the property that $d(x,y) = 0 \iff x = y$. Also, we have
    \[\mathbbm{1}\{x_i \neq y_i\} \leq \mathbbm{1}\{x_i \neq z_i\} + \mathbbm{1}\{z_i \neq y_i\}\]
    This is easy to see using a case-by-case analysis. 
    \begin{enumerate}
        \item If $x_i = y_i$, then $\mathbbm{1}\{x_i \neq y_i\} = 0$. If $z_i = x_i = y_i$, we have $\mathbbm{1}\{x_i \neq z_i\} + \mathbbm{1}\{z_i \neq y_i\} = 0$, while if $z_i \neq x_i = y_i$, then $\mathbbm{1}\{x_i \neq z_i\} + \mathbbm{1}\{z_i \neq y_i\} = 2$, and hence, $\mathbbm{1}\{x_i \neq y_i\} \leq \mathbbm{1}\{x_i \neq z_i\} + \mathbbm{1}\{z_i \neq y_i\}$.
        \item If $x_i \neq y_i$, then $\mathbbm{1}\{x_i \neq y_i\} = 1$. If $z_i = x_i$ or $z_i = y_i$, we have $\mathbbm{1}\{x_i \neq z_i\} + \mathbbm{1}\{z_i \neq y_i\} = 1$, while if $z_i \neq x_i \neq y_i$, then $\mathbbm{1}\{x_i \neq z_i\} + \mathbbm{1}\{z_i \neq y_i\} = 2$, and hence, $\mathbbm{1}\{x_i \neq y_i\} \leq \mathbbm{1}\{x_i \neq z_i\} + \mathbbm{1}\{z_i \neq y_i\}$.
    \end{enumerate}
    Thus, the Hamming distance satisfies all the properties of a metric.
\end{proof}

\begin{question}
    Prove (1).
    \label{section1.1-11}
\end{question}
\begin{proof}
    We wish to prove the statement:
    \[d(x_1 , x_n) \leq  \sum_{i=2}^n d(x_{i-1} , x_i)\]
    We can do this by induction on $n$. The base case $n = 3$ holds due to the triangle inequality. Assuming it holds for $n = k$, we wish to show that it holds for $n = k+1$. This is easy to show as follows:
    \[d(x_1 , x_{k+1}) \leq d(x_1 , x_k) + d(x_k , x_{k+1}) \leq \sum_{i=2}^k d(x_{i-1} , x_{i}) + d(x_k , x_{k+1}) = \sum_{i=2}^{k+1} d(x_{i-1} , x_i)\]
    where the first inequality follows by the triangle inequality and the second follows from the induction hypothesis.
\end{proof}

\begin{question}
    Using (1), show that
    \[\abs{d(x,y) - d(z,w)} \leq d(x,z) + d(y,w)\]
    \label{section1.1-12}
\end{question}
\begin{proof}
    By the repeated application of triangle inequality, we have
    \[d(x,y) \leq d(x,z) + d(z,y) \leq d(x,z) + d(z,w) + d(w,y)\]
    which gives us 
    \[d(x,y) - d(z,w) \leq d(x,z) + d(y,w)\]
    Similarly, we have
    \[d(z,w) \leq d(z,x) + d(x,z) \leq d(z,x) +  d(x,y) + d(y,w)\]
    which gives us
    \[d(z,w) - d(x,y) \leq d(x,z) + d(y,w)\]
    Combining both these facts and using the symmetry property of metrics gives us the required result.
\end{proof}

\begin{question}
    Using the triangle inequality, show that
    \[\abs{d(x,z) - d(y,z)} \leq d(x,y)\]
    \label{section1.1-13}
\end{question}
\begin{proof}
   Note that $d(x,z) \leq d(x,y) + d(y,z)$ and hence, $d(x,z) - d(y,z) \leq d(x,y)$. Similarly, $d(y,z) \leq d(x,y) + d(x,z)$ and hence, $d(y,z) - d(x,z) \leq d(x,y)$. Combining both results finishes the proof.
\end{proof}

\begin{question}
    Show that (M3) and (M4) could be obtained from (M2) and 
    \[d(x,y) \leq d(z,x) + d(z,y)\]
    \label{section1.1-14}
\end{question}
\begin{proof}
    We wish to show that the axioms $d(x,y) = d(y,x)$ and $d(x,y) \leq d(x,z) + d(z,y)$ can be obtained from the axioms $d(x,y) = 0 \iff x = y$ and $d(x,y) \leq d(z,x) + d(z,y)$.

    To prove symmetry, we have the following two equations:
    \[d(x,y) \leq d(z,x) + d(z,y)\]
    \[d(y,x) \leq d(z,y) + d(z,x)\]
    Subtracting both equations results in
    \[d(x,y) \leq d(y,x) \text{ and } d(y,x) \leq d(x,y) \implies d(x,y) = d(y,x).\]
    Now, simply using symmetry, we have
    \[d(x,y) \leq d(z,x) + d(z,y) = d(x,z) + d(z,y)\]
    since $d(z,x) = d(x,z)$.
\end{proof}

\begin{question}
    Show that non-negativity of a metric follows from (M2) to (M4).
    \label{section1.1-15}
\end{question}
\begin{proof}
    We wish to show that the non-negativity of a metric follows from the following axioms:
    \begin{enumerate}
        \item $d(x,y) = d(y,x)$.
        \item $d(x,y) = 0 \iff x = y$.
        \item $d(x,y) \leq d(x,z) + d(z,y)$.
    \end{enumerate}
    Substituting $y = x$ in the triangle inequality, we have
    \[d(x,x) \leq d(x,z) + d(z,x)\]
    Using the first and second axiom in tandem gives us $d(x,z) \geq 0$, which completes the proof.
\end{proof}