\subsection{Completion of Metric Spaces}

\begin{question}
    Show that if a subspace $Y$ of a metric space consists of finitely many points, then $Y$ is complete.
    \label{section1.6-1}
\end{question}
\begin{proof}
    First, suppose there is a Cauchy sequence $x_n$. This means that for every $\epsilon \g 0$, there exists $N(\epsilon)$ and some $m,n \g N(\epsilon)$ such that $d(x-m , x_n) \l \epsilon$. However, due to a finite number of points available in the set, the sequence has finitely many terms, and thus, we can express $\epsilon_0 = \min_{a , b \in Y} d(a,b)$. Thus, if we choose $\epsilon = \epsilon_0$, the sequence will have to become constant, which always converges. Thus, all Cauchy sequences become constant eventually, which converge, and hence, the space is complete.
\end{proof}

\begin{question}
    What is the completion of $(X,d)$ where $X$ is the set of all rational numbers and $d(x,y) = \abs{x-y}$?
    \label{section1.6-2}
\end{question}
\begin{proof}
    The completion would be the set of all reals, $\R$.
\end{proof}

\begin{question}
    What is the completion of a discrete metric space?
    \label{section1.6-3}
\end{question}
\begin{proof}
    Recall that a discrete metric space is always complete. Hence, the completion is the entire space $X$ itself.
\end{proof}

\begin{question}
    If $X_1$ and $X_2$ are isometric and $X_1$ is complete, show that $X_2$ is complete.
    \label{section1.6-4}
\end{question}
\begin{proof}
    $X_1$ and $X_2$ are isometric spaces if there exists some bijective isometry $T : X_1 \mapsto X_2$. By an isometry, we mean that for some $x , y \in X_1$
    \[d_1(x,y) = d_2(Tx , Ty).\]

    If $X_1$ is complete, then $X_1$ is also closed, i.e, $\overline{X_1} = X_1$. Suppose $x \in X_1$ is an accumulation point of $X_1$. This means that for every $\epsilon\g 0$, there exists some $x_\epsilon \in X_1$ such that 
    \[d_1(x , x_\epsilon) \l \epsilon.\]
    Now, since $X_1$ and $X_2$ are isometric spaces
    \[d_2(Tx , Tx_\epsilon) \l \epsilon\]
    which means that $Tx \in X_2$ is also an accumulation point for $X_2$ (since $x_\epsilon$ is arbitrary). Thus, all accumulation points of $X_2$ lie in $X_2$ and hence, $X_2$ is also complete.
\end{proof}

\begin{question}
    A homeomorphism is a continuous bijective mapping $T : X \mapsto Y$ whose inverse is continuous. The metric spaces $X$ and $Y$ are said to homeomorphic. Show that is $X$ and $Y$ are isometric, then they are also homeomorphic.
\label{section1.6-5}
\end{question}
\begin{proof}
    First, an isometry implies the existence of a bijective map, so we already have a bijection. We need to ensure both the map and it's inverse is continuous. The definition of continuity is as follows: for every $\epsilon \g 0$, there exists $\delta \g 0$, such that $d_1(x , x_0) \l \delta \implies d_2(Tx , Tx_0) \l \epsilon$ ensures that the map $T$ is continuous as $x_0$. Now, from the condition of isometric spaces, we have that
    \[d_1(x , x_0) = d_2(Tx ,Tx_0)\]
    and hence, if we set $\delta = \epsilon$, we see that the definition of continuity is always fulfilled. Thus, the mapping from $X$ to $Y$ is always continuous. Making a similar argument for $T^{-1}$ results in the conclusion that $T^{-1}$ is continuous and hence, $X$ and $Y$ are also homeomorphic.
\end{proof}

\begin{question}
    Show that $C[0,1]$ and $C[a,b]$ are isomorphic.
    \label{section1.6-6}
\end{question}
\begin{proof}
    To show that $C[0,1]$ and $C[a,b]$ is isometric, we wish to find some mapping $T : C[0,1] \mapsto C[a,b]$ such that for some $x , y \in C[0,1]$, 
    \[d_1(x,y) = d_2(Tx , Ty),\]
    where $d_1$ and $d_2$ are metrics on $C[0,1]$ and $C[a,b]$ respectively.
    
    Let $x , y \in C[0,1]$. Let $T : C[0,1] \mapsto C[a,b]$. Then, we can define $Tx \in C[a,b]$ as
    \[(Tx) (t) = x\left( \frac{t-a}{b-a}\right).\]
    In such a case,
    \[d(Tx , Ty)  = \max_{t \in [a,b]} \abs{(Tx)(t) - (Ty)(t)} = \max_{t \in [a,b]} \abs{x\left( \frac{t-a}{b-a}\right) - y\left( \frac{t-a}{b-a}\right)} = \max_{t \in [0,1]} \abs{x(t) - y(t)} = d(x,y)\]
    and hence, the map is distance preserving. We can similarly define for $x , y \in C[a,b]$, $T^{-1} : C[a,b] \mapsto C[0,1]$ such that $T^{-1}x \in C[0,1]$
    \[(T^{-1}x)(t) = x((b-a)t + a)\]
     and hence, we have
     \[d(T^{-1}(x) , T^{-1}y) = \max_{t \in [0,1]} \abs{(T^{-1}x)(t) - (T^{-1}y)(t)}  = \max_{t \in [a,b]} \abs{x(t) - y(t)} = d(x,y)\]
      and hence, $T^{-1}$ is also distance preserving. Further, because of the linear mapping, the mapping is bijective. Hence, the spaces are isometric.
\end{proof}

\begin{question}
    If $(X,d)$ is complete, show that $(X, \tilde{d})$ where $\tilde{d} = \frac{d}{1+d}$ is complete.
    \label{section1.6-7}
\end{question}
\begin{proof}
    Let $x_m$ be some Cauchy sequence in the space $(X,d)$, such that for every $\epsilon \g 0$, there exists some $N(\epsilon)$, such that for all $m , n \g N(\epsilon)$, $d(x_m  ,x_n) \l \epsilon.$ Now, consider the same sequence under the metric $\tilde{d}$. Thus, we have
    \[\tilde{d}(x_m , x_n) = \frac{d(x_m,x_n)}{1 + d(x_m , x_n)} \l \epsilon\]
    and hence, $x_m$ is a Cauchy sequence in $(X,\tilde{d})$ too. Since $(X,d)$ is complete, we have that every Cauchy sequence converges to some point in $X$. Let $x_m \rightarrow x$, then we have that for some $\epsilon \g 0$, there exists $N(\epsilon)$ such that for $n \g N(\epsilon)$, $d(x_n,x) \l \epsilon.$ Similarly, we have for the same sequence $x_n$, 
    \[\tilde{d}(x , x_n) = \frac{d(x,x_n)}{1 + d(x , x_n)} \l \epsilon\]
    and hence, the sequence converges in $(X,\tilde{d})$. Thus, this space is also complete.
\end{proof}

\begin{question}
    Show that in \ref{section1.6-7}, completeness of $(X,\tilde{d})$ implies completeness of $(X,d)$.
    \label{section1.6-8}
\end{question}
\begin{proof}
    Since $(X,  \tilde{d})$ is complete, we have that all accumulation points of $X$ belong to $X$ itself. Let $x \in X$ be an accumulation point. Then, for every $\epsilon \g 0$, there exists $x_\epsilon$ such that
    \[\tilde{d}(x , x_\epsilon) \l \epsilon \implies \frac{d(x,  x_\epsilon)}{1 + d(x , x_\epsilon )} \l \epsilon \implies d(x , x_\epsilon) \l \frac{\epsilon}{1 - \epsilon}\]
    Thus, if $\epsilon \l 1$, $x_\epsilon$ is also an accumulation point for $X$ under the metric $d$. On the other hand, if $\epsilon \g 1$, then this is trivially true.
\end{proof}

\begin{question}
    If $(x_n)$ and $(x_n^\prime)$ om $(X,d)$  are such that $(1)$ holds and $x_n \rightarrow l$, show that $(x_n^\prime)$ converges and has the limit $l$.
    \label{section1.6-9}
\end{question}
\begin{proof}
    (1) is the following: 
    \[\lim_{n \rightarrow \infty} d(x_n,x_n^\prime) = 0\]
    Since $x_n$ converges to $l$, we have that
    \[\lim_{n \rightarrow \infty} d(x_n,l) = 0\]
    Thus, we have
    \[0 \leq \lim_{n \rightarrow \infty} d(x_n^\prime , l) \leq \lim_{n \rightarrow \infty} d(x_n^\prime , x_n) + \lim_{n \rightarrow \infty} d(x_n , l) = 0\]
    and hence, $x_n^\prime$ converges to $l$.
\end{proof}

\begin{question}
    If $(x_n)$ and $(x_n^\prime)$ are convergent sequences in a metric space $(X,d)$ and have the same limit $l$, show that the satisfy (1).
    \label{section1.6-10}
\end{question}
\begin{proof}
     (1) is the following: 
    \[\lim_{n \rightarrow \infty} d(x_n,x_n^\prime) = 0\]
    Since, both $(x_n)$ and $(x_n^\prime)$ converge to the same limit $l$, we have that
    \[\lim_{n \rightarrow \infty} d(x_n,l) = 0\]
    \[\lim_{n \rightarrow \infty} d(x_n^\prime,l) = 0\]
    Thus, we have
    \[0 \leq \lim_{n \rightarrow \infty} d(x_n,x_n^\prime)  \leq \lim_{n \rightarrow \infty} d(x_n,l) + \lim_{n \rightarrow \infty} d(l,x_n^\prime) = 0\]
    This finishes the proof.
\end{proof}

\begin{question}
    Show that (1) defines an equivalence relation on the set of all Cauchy sequences of elements of $X$.
    \label{section1.6-11}
\end{question}
\begin{proof}
    (1) claims that two Cauchy sequences $(x_n)$ and $(x_n^\prime)$ are equivalent (represented by $x_n \sim x_n^\prime$) if 
    \[\lim_{n \rightarrow \infty} d(x_n , x_n^\prime) = 0\]
    An equivalence relation $R \subset X \times X$ satisfies the following three claims:
    \begin{enumerate}
        \item $R(x,x) \forall x \in X$.
        \item $R(x,y) = R(y,x) \forall x, y \in X$.
        \item $R(x,y)$ and $R(y,z)$ implies $R(x,y)$ $\forall x,y,z \in X$.
    \end{enumerate}
    Clearly, $\lim_{n \rightarrow \infty} d(x_n , x_n) = 0$ and also $\lim_{n \rightarrow \infty} d(x_n , x_n^\prime) = \lim_{n \rightarrow \infty} d(x_n^\prime , x_n) = 0$. Suppose, $\lim_{n \rightarrow \infty} d(x_n , z_n) = 0$ and $\lim_{n \rightarrow \infty} d(y_n , z_n) = 0$, then
    \[0 \leq \lim_{n \rightarrow \infty} d(x_n , y_n) \leq \lim_{n \rightarrow \infty} d(x_n , z_n) + \lim_{n \rightarrow \infty} d(z_n , y_n) = 0 \implies \lim_{n \rightarrow \infty} d(x_n , y_n) = 0\]
    Thus, the relation satisfies all the conditions of being an equivalence relation.
\end{proof}

\begin{question}
    If $(x_n)$ is Cauchy in $(X,d)$ and $(x_n^\prime)$ in $X$ satisfies (1), show that $(x_n^\prime)$ is Cauchy in $X$.
    \label{section1.6-12}
\end{question}
\begin{proof}
    Since $x_n$ is Cauchy in $X$, we have that for every $\epsilon \g 0$, there exists $N(\epsilon)$ such that for all $m,n \g N(\epsilon)$, $D(x_m , x_n) \l\epsilon.$ Also, (1) claims that
    \[\lim_{n \rightarrow \infty} d(x_n , x_n^\prime) = 0.\]
    Thus, for some $\epsilon \g 0$ and $m,n \g N(\epsilon)$, we have
    \[d(x_n^\prime , x_m^\prime) \leq d(x_n^\prime , x_n) + d(x_n , x_m) + d(x_m^\prime , x_m) \leq 3\epsilon.\]
    Thus, $(x_n^\prime)$ is also Cauchy.
\end{proof}

\begin{question}
    A finite psuedometric on a set $X$ is a function $d : X \times X \mapsto \R$ satisfying (M1) , (M3), (M4) and
    \[\text{(M2*)} d(x,x) = 0.\]
    What is the difference between a metric and pseudfometric. Show that $d(x,y) = \abs{\xi_1 - \eta_1}$ defines a pseudometric on the set of all ordered pairs of real numbers where $x = (\xi_1 , \xi_2)$ and $y = (\eta_1 , \eta_2).$
    \label{section1.6-13}
\end{question}
\begin{proof}
    Let us reinterpret the original condition M2 for a metric
    \[d(x,y) = 0 \iff x = y\]
    This is a two sided condition, i.e if $d(x,y) = 0$, then $x = y$. On the other hand, if $x = y$, then $d(x,y) = 0$. However the condition for a psuedometric seems to be one sided, in other words the distance between the same point is $0$. However, if the distance between two points is $0$, it does not necessarily mean it is the same point. In other words, $x = y \implies d(x,y) = 0$. However, $d(x,y) = 0 \nRightarrow x = y$. This is clear from the example. Suppose $x = y = (\xi_1 , \xi_2)$. Then, $d(x,y) = \abs{\xi_1 - \xi_1} = 0$. However, $d(x,y) = \abs{\xi_1 - \eta_1} = 0 \implies \xi_1 = \eta_1$. However, this does not mean $\xi_2 = \eta_2$ and hence, $x$ and $y$ may not be equal points.
\end{proof}

\begin{question}
    Does 
    \[d(x,y) = \int_{a}^{b} \abs{x(t) - y(t)} \; dt\]
    define a metric or pseudometric on $X$ if $X$ is the set of all real-valued continuous functions on $[a,b]$ and if $X$ is the set of all real-values Riemann integrable functions on $[a,b]$?
    \label{section1.6-14}
\end{question}
\begin{proof}
    If $X$ is the set of all real-valued continuous functions, then it is evident that if $x = y$, then $d(x,y) = 0$. However, if $d(x,y) = 0$, since the metric is the integral of positive quantities, the only way the sum is zero is if each of the quantities is zero, i.e $x(t) = y(t)$ for all $t \in [a,b]$.

    On the other hand, if $X$ is the set of all real-valued Riemann integrable functions, then it is once again clear that if $x = y$, $d(x,y) = 0$. However, $d(x,y) = 0$ does not imply $x = y$. For example, consider the function $x(t) = c_1$ and $y(t) = c_2$ if $t \in C$ where $C$ is a finite set, and $c_1$ otherwise. Then, the pointwise difference does not matter, and the integral shall remain $0$.
\end{proof}

\begin{question}
    If $(X,d)$ is a pseudometric space, we call a set
    \[B(x_0 , r) = \{x \in X \mid d(x,x_0) \l r\}\]
    an open ball in $X$ with center $x_0$ and radius r. What are open balls of radius 1 in \ref{section1.6-13}?
    \label{section1.6-15}
\end{question}
\begin{proof}
    In problem 13, we have $X$ to  be the set of ordered real value pairs such that $x = (\xi_1 , \xi_2)$ and $y = (\eta_1 , \eta_2)$, then, $d(x,y) = \abs{\xi_1 - \eta_1}.$ Suppose, $x_0 = (a , b)$. Then, any point within distance $1$ of $x_0$ has first coordinate $a \pm 1$, while the second coordinate does not matter. Thus, the open ball would be of the form
    \[B((a,b) , 1) = \{(a \pm 1 , r) \mid r \in \R\}\]
    which results in vertical strips of width 2 centered at $a$.
\end{proof}