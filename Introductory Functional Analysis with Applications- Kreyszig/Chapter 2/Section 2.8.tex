\subsection{Linear Functionals}

\begin{question}
    Show that the functionals in 2.8-7 and 2.8-8 are linear.
    \label{section2.8-1}
\end{question}
\begin{proof}
    In 2.8-7, we have the functional
    \[f_1(x) = x(t_0)\]
    over the space $C[a,b]$ and for a fixed $t_0 \in [a,b]$. Note that
    \[f(\alpha x + \beta y) = (\alpha x + \beta y)(t_0) = \alpha x(t_0) + \beta y(t_0) = \alpha f(x) + \beta f(y)\]
    and hence, $f$ is linear.

    In 2.8-8, we have the functional 
    \[f(x) = \sum_{j=1}^\infty \xi_j a_j\]
    for $x = (\xi_j) \in \ell^2$ and $a = (a_j) \in \ell^2.$
    Note that
    \[f(\alpha x + \beta y) = \sum_{j=1}^\infty a_j(\alpha \xi_j + \beta \eta_j) = \alpha \sum_{j=1}^\infty  a_j \xi_j + \beta \sum_{j=1}^\infty a_j \eta_j = \alpha f(x) + \beta f(y) \]
    and hence, $f$ is linear.
\end{proof}

\begin{question}
    Show that the functionals defined on $C[a,b]$ by
    \[f_1(x) = \int_{a}^{b} x(t) y_0(t) \;dt \;\; y_0 \in C[a,b]\]
    \[f_2(x) = \alpha x(a)+ \beta x(b)\]
    are linear and bounded.
    \label{section2.8-2}
\end{question}
\begin{proof}
    First, we have
    \[f_1(\alpha x + \beta y) = \int_{a}^{b} (\alpha x + \beta y)(t) y_0(t) \; dt = \alpha \int_{a}^{b} x(t) y_0(t) \; dt + \beta \int_{a}^{b} y(t)y_0(t) \; dt = \alpha f_1(x) + \beta f_1(y)\]
    and hence, $f_1$ is linear. We now show $f_1$ is bounded as follows:
    \[\abs{f_1(x)} = \abs{\int_{a}^{b} x(t) y_0(t) \; dt} \leq \int_{a}^{b} \abs{x(t)}\abs{y_t(t)} \;dt \leq \int_{a}^{b} \left( \max_{t \in [a,b]} \abs{x(t)}  \right) \abs{y_0(t)} \;dt = \norm{x} \int_{a}^{b} \abs{y_0}(t) \; dt\]
    and hence, we have
    \[\norm{f} \leq \int_{a}^{b} \abs{y_0(t)} \; dt \]

    Now, we have
    \[f_2(\gamma x + \delta y) = \alpha (\gamma x + \delta y)(a) + \beta (\gamma x + \delta y)(b) = \gamma (\alpha x(a) + \beta x(b)) + \delta (\alpha y(a) + \beta y(b)) = \gamma f_2(x) + \delta f_2(y), \]
    and hence, $f_2$ is linear. To show $f_2$ is bounded, we have
    \[\abs{f_2(x)} = \abs{\alpha x(a) + \beta x(b)} \leq \abs{\alpha} \max_{t \in [a,b]} \abs{x(t)} + \abs{\beta} \max_{t \in [a,b]} \abs{x(t)} = (\abs{\alpha} + \abs{\beta}) \norm{x}\]
    and hence, 
    \[\norm{f_2} \leq (\abs{\alpha} + \abs{\beta}).\]
\end{proof}

\begin{question}
    Find the norm of the linear functional $f$ defined on $C[-1,1]$ by
    \[f(x) = \int_{-1}^{0} x(t) \; dt - \int_{0}^{1} x(t) \;dt \]
    \label{section2.8-3}
\end{question}
\begin{proof}
    Recall the definition of the norm:
    \[\norm{f} = \sup_{\norm{x} = 1} \abs{f(x)}.\]
    Consider some $x$ with norm $1$, 
    \[\norm{x} = 1 \implies \max_{t \in [-1,1]} \abs{x(t)} = 1 \implies \abs{x(t)} \leq 1 \; \forall t \in [-1,1].\]
    Now,
    \[\norm{f} = \sup_{\norm{x} = 1} \abs{f(x)} = \sup_{\norm{x} = 1} \abs{\int_{-1}^{0} x(t) \; dt - \int_{0}^{1} x(t) \; dt} \leq \sup_{\norm{x} = 1} \int_{-1}^{0} \abs{x(t)} \; dt + \sup_{\norm{x} = 1} \int_{0}^{1} \abs{x(t)} \; dt \leq 2.\]
    On the other hand, using $\abs{f(x)} \leq \norm{f} \norm{x}$, for $x(t) = 1 \; \forall t \in [-1,0]$ and $x(t) = -1 \;\forall t \in [0,1]$ and hence, $\norm{x} = 1$, we get
    \[\norm{f} \geq \frac{\abs{f(x)}}{\norm{x}} = \abs{f(x)} = \abs{\int_{-1}^{0} 1 \;dt - \int_{0}^{1} -1 \; dt} = 2\]
    and hence, $\norm{f} = 2$.
\end{proof}

\begin{question}
    Show that 
    \[f_1(x) = \max_{t \in [a,b]} x(t)\]
    \[f_2(x) = \min_{t \in [a,b]} x(t)\]
    define functionals on $C[a,b]$. Are they linear and bounded?
    \label{section2.8-4}
\end{question}
\begin{proof}
    Note that both $f_1$ and $f_2$ are functionals since they take as input an element of the space, in this case, $C[a,b]$, and return the pointwise maximum (or minimum), which is an element of the underlying scalar field.

    Now, 
    \[f_1(\alpha x + \beta y) = \max_{t \in [a,b]} (\alpha x + \beta y)(t) \leq \alpha \max_{t \in [a,b]} x(t) + \beta \max_{t \in [a,b]} y(t) = \alpha f_1(x) + \beta f_2(x)\]
     and hence, the functional is not linear. Now, we have
     \[\abs{f(x)} = \abs{\max_{t \in [a,b]} x(t)} \leq \max_{t \in [a,b]} \abs{x(t)} = 1 \cdot \norm{x}, \]
     and hence $\norm{f_1} \leq 1$. Now, choose $x(t) = 1 \;\forall t\in [a,b]$, then $\norm{x} = 1$ and 
     \[\norm{f_1} \geq \frac{\abs{f_1(x)}}{\norm{x}} = \frac{\abs{\max_{t \in [a,b]}x(t)}} {\norm{x}} = 1,\]
     and hence, $\norm{f_1} = 1$. So, $f_1$ is bounded. 

     Similarly, we have
     \[f_2(\alpha x + \beta y) \geq \alpha f_2(x) + \beta f_2(y)\]
      and hence, $f_2$ is not linear. Now, we have
      \[\abs{f_2(x)} = \abs{\min_{t \in [a,b]} x(t)} \leq \abs{\max_{t \in [a,b]} x(t)} \leq \max_{t \in [a,b]} \abs{x(t)} \leq 1 \cdot \norm{x} \]
      and hence, $f_2$ is bounded.
\end{proof}

\begin{question}
    Show that on any sequence space $X$ we can define a linear functional $f$ by setting $f(x) = \xi_n$ ($n$ fixed), where $(\xi_j)$. Is $f$ bounded if $X = \ell^\infty$?
    \label{section2.8-5}
\end{question}
\begin{proof}
    First, $f(x) = \xi_n$ is a functional since it takes as input a sequence $x$ and outputs an element from the sequence, which belongs to the underlying scalar field. Also,
    \[f(\alpha x + \beta y) = (\alpha x + \beta y)_n = \alpha \xi_n + \beta \eta_n = \alpha f(x) + \beta f(y).\]
    Hence, it is linear. Also, if $x \in \ell^\infty$, we have
    \[\abs{f(x)} = \abs{\xi_n} \leq \max_{j} \abs{\xi_j} \leq 1\cdot \norm{x}_\infty\]
    and hence, $f$ is bounded.
\end{proof}

\begin{question}
    The space $C^1[a,b]$ or $C^\prime[a,b]$ is the normed space of all continuously differentiable functions on $J = [a,b]$ with norm defined by
    \[\norm{x} = \max_{t \in [a,b]} \abs{x(t)} + \max_{t \in [a,b]} \abs{x^\prime(t)}.\]
    Show that the axioms of a norm are satisfied. Show that $f(x) = x^\prime (c), c = (a+b)/2$, defines a bounded linear functional on $C^\prime[a,b]$. Show that $f$ is not bounded, considered as a functional on the subspace of $C[a,b]$ which consists of all continuously differentiable functions.
    \label{section2.8-6}
\end{question}
\begin{proof}
    First, $\norm{x}$ is clearly non-negative. Also, if $x = 0$, then $\norm{x} = 0$. However, if $\norm{x} = 0$, then 
    \[\max_{t \in [a,b]} \abs{x(t)} = 0 \implies x(t) = 0 \; \forall t \in [a,b]\]
    \[\max_{t \in [a,b]} \abs{x^\prime(t)} = 0 \implies x^\prime(t) = \;\forall t\in [a,b] \implies x(t) = k \;\forall t \in [a,b]\]
    Hence, $\norm{x} = 0 \implies x = 0$.
    Now, we have
    \[\norm{\alpha x} = \max_{t \in [a,b]}\abs{\alpha x(t)} + \max_{t \in [a,b]} \abs{\alpha x^\prime (t)} = \abs{\alpha} \norm{x}.\]
    And finally, we have
    \[\norm{x + y} = \max_{t \in [a,b]} \abs{x(t) + y(t)} + \max_{t \in [a,b]} \abs{x^\prime(t) + y^\prime (t)} \leq \max_{t \in [a,b]} \abs{x(t)} + \max_{t\in [a,b]} \abs{x^\prime(t)} + \max_{t \in [a,b]} \abs{y(t)} + \max_{t \in [a,b]} \abs{y^\prime(t)} = \norm{x} + \norm{y}.\]

    Thus, $\norm{x}$ defines a valid norm. Now, define $f(x) = x^\prime(c) , c = (a+b)/2$. Then, we have
    \[f(\alpha x + \beta y) = (\alpha x + \beta y)^\prime(c) = \alpha x^\prime(c) + \beta y^\prime(c) = \alpha f(x) + \beta f(y)\]
    and hence, $f$ is linear. Also, 
    \[\abs{f(x)} = \abs{x^\prime(c)} \leq \max_{t \in [a,b]} \abs{x^\prime(t)} \leq \max_{t \in [a,b]} \abs{x(t)} + \max_{t \in [a,b]} \abs{x^\prime(c)} = \norm{x}.\]
    Hence, $f$ is also bounded.

    Now, consider $f$ on the subspace of $C[a,b]$ with all continuously differentiable functions. Here, the norm is defined as 
    \[\norm{x} = \sup_{t \in [a,b]} \abs{x(t)}.\]
    Suppose $\norm{x} = 1 \implies \abs{x(t)} \leq 1 \;\forall t \in [a,b]$. We wish to find some $x$ such that 
    \[x^\prime((a+b)/2)\geq 1.\]
    \textcolor{red}{TODO.}
\end{proof}

\begin{question}
    If $f$ is a bounded linear functional on a complex normed space, is $\overline{f}$ linear and bounded? Here $\overline{}$ defined complex conjugate.
    \label{section2.8-7}
\end{question}
\begin{proof}
    Consider $\overline{f}(\alpha x) = \overline{f(\alpha x} = \overline{\alpha f(x)} = \overline{\alpha} \overline{f}(x).$ Hence, it is not linear. However, $\abs{\overline{f}(x)} = \abs{\overline{f(x)}} = \abs{f(x)} \leq \norm{f}\norm{x}$. Since $f$ is bounded, $\overline{f}$ is also bounded.
\end{proof}

\begin{question}
    The null space $N(M^\star)$ of a set $M^\star \subset X^\star$ is defined to be the set of all $x \in X$ such that $f(x) = 0$ for all $f \in M^\star$. Show that $N(M^\star)$ is a vector space.
    \label{section2.8-8}
\end{question}
\begin{proof}
    Suppose $x_1 , x_2 \in N(M^\star).$ Then, $f(x_1) = 0$ and $f(x_2) = 0$ for all $f \in M^\star$. Then, we have that for any $f \in M^\star$
    \[f(\alpha x_1 + \beta x_2) = \alpha f(x_1) + \beta f(x_2) = 0\]
    and hence, $\alpha x_1 + \beta x_2 \in N(M^\star)$. The equality follows since by definition, $X^\star$ contains the linear functionals. Hence, $N(M^\star)$ is a vector space.
\end{proof}

\begin{question}
    Let $f\neq 0$ be any linear functional on a vector space $X$ and $x_0$ any fixed element of $X - \nul{f}$ where $\nul{f}$ is the null space of $f$. Show that any $x \in X$ has a unique representation $x = \alpha x_0 + y$, where $y \in \nul{f}$.
    \label{section2.8-9}
\end{question}
\begin{proof}
    We have that $y = x - ax_0$. Thus, 
    \[f(y) = f(x) - f(ax_0) = f(x) - af(x_0)\]
    Thus, setting $a = f(x) / f(x_0)$ results in $f(y) = 0 \implies y \in \nul{F}$. To prove that it is unique, let $x = \alpha_1 x_0 + y_1 = \alpha_2 x_0 + y_2$. Then, 
    \[f(x) = \alpha_1 f(x_0) = \alpha_2 f(x_0) \]
    since $f(y_1) = f(y_2) = 0$. Now, since $x_0 \notin \nul{f}$, we have $\alpha_1 = \alpha_2$ which results in $y_1 = y_2.$
\end{proof}

\begin{question}
    Show that in \ref{section2.8-9}, two elements $x_1 , x_2 \in X$ belong to the same element of the quotient space $X/\nul{f}$ if and only if $f(x_1) = f(x_2)$. Show that $\textrm{codim} \; \nul{f} = 1$.
    \label{section2.8-10}
\end{question}
\begin{proof}
    Suppose $x_1 , x_2 \in X$ belong to the same element of the quotient space $X/\nul{f}$. Then, there exists some $x \in X$ such that for some $y_1 , y_2 \in \nul{f}$
    \[x_1 = x + y_1 \;,\; x_2 = x + y_2.\]
    Then, 
    \[f(x_1) = f(x + y_1) = f(x) + f(y_1) = f(x).\]
    \[f(x_2) = f(x + y_2) = f(x) + f(y_2) = f(x).\]
    Thus, $f(x_1) = f(x_2).$
    
    Now, suppose $f(x_1) = f(x_2).$ Then, 
    \[f(x_1) - f(x_2) = 0 = f(x_1 - x_2) \implies x_1 - x_2 \in \nul{f}\]
    Thus, it easy to see that if $x_1 - x_2 \in \nul{f}$, then $x_1$ and $x_2$ differ by an additive factor of some $y$, $y \in \nul{f}$, and hence, they belong to the same coset.

    Since we can write any $x = \alpha x_0 + y = \alpha (x_0 + \frac{1}{\alpha} y)$, $y \in \nul{f}$, the codimension of the cosets is $1$.
\end{proof}

\begin{question}
    Show that two linear functionals $f_1 \neq 0$ and $f_2 \neq 0$ which are defined on the same vector space and have the same null space are proportional.
    \label{section2.8-11}
\end{question}
\begin{proof}
    First, note that any element in the kernel space, say $x \in \nul{f_1}$ can be written as $x_1 + \alpha x_2$ where $\alpha = -\frac{f_1(x_1)}{f_1(x_2)}$. Since $\nul{f_1} = \nul{f_2}$, $x \in \nul{f_2}$, and hence, we can also write $x = x_1 + \alpha x_2$ where $\alpha = -\frac{f_2(x_1)}{f_2(x_2)}$. Thus, we get, for some constant $k$
    \[-\frac{f_1(x_1)}{f_1(x_2)} = -\frac{f_2(x_1)}{f_2(x_2)} = k \implies f_1(x) = k f_2(x) \;\forall x \in X\]
    Hence, the functions are proportional.
\end{proof}

\begin{question}
    If $Y$ is a subspace of a vector space $X$ and $\textrm{codim}\;Y= 1$, then every element of $X/Y$ is called a hyperplane parallel to $Y$. Show that for any linear functional $f \neq 0$ on $X$, the set $H_1 = \{x \in X \mid f(x) = 1\}$ is a hyper plane parallel to the null space $\nul{f}$ of $f$.
    \label{section2.8-12}
\end{question}
\begin{proof}
    We wish to show that $H_1 = \{x \in X \mid f(x) = 1\}$ is an element of $X/\nul{f}$ and that $\textrm{codim}\; \nul{f} = 1$. We have shown that the codimension of the null space is $1$ in \ref{section2.8-10}. Now, we show that $H_1$ is an element of $X/\nul{f}$. Let $x = \alpha x_0 + y$ for some fixed $x_0$, $y \in \nul{f}$, and $f(x) = 1$. Then, $\alpha = \frac{1}{f(x_0)}$. Note that $f(\alpha x_0) = 1$. Hence, we have expressed $H_1 = \{\alpha x_0 + y \mid y \in \nul{f}\}$ and hence, it is coset parallel to the null space (shifted by $\alpha x_0$.)
\end{proof}

\begin{question}
    If $Y$ is a subspace of  a vector space $X$ and $f$ is a linear functional on $X$ such that $f(Y)$ is not the whole scalar field of $X$, show that $f(y) = 0 \; \forall y \in Y$.
    \label{section2.8-13}
\end{question}
\begin{proof}
    Since $Y$ is a subspace, the zero element belongs to $Y$. Hence, $0$ has to be one of the values in the range of $f$. Suppose $f(y_1) = k \neq 0$. Then, for any $k_0 \in K$, we have that 
    \[f\left(\frac{k_0}{k} y\right) = k_0 \]
    or, in other words, $k_0$ will also belong to the range of $f$. Since $k_0$ was arbitrary, the entire scalar field will be the range of $f$. Hence, $f(y) = 0\;\forall y \in Y$ so that the complete scalar field is not the range.
\end{proof}

\begin{question}
    Show that the norm $\norm{f}$ of a bounded linear functional $f \neq 0$ on a normed space $X$ can be interpreted geometrically as the reciprocal of the distance $\tilde{d} = \inf\{\norm{x} \mid f(x) = 1\}$ of the hyperplane $H_1 = \{x \in X \mid f(x) = 1\}$ from the origin.
    \label{section2.8-14}
\end{question}
\begin{proof}
    We know that 
    \[\norm{f} = \sup_{x \neq 0} \frac{\abs{f(x)}}{\norm{x}}.\]
    Suppose we condition on the hyperplane $\{x \mid f(x) = 1\}$. Then, 
    \[\norm{f} = \sup_{x : f(x) = 1 , x\neq 0} \frac{1}{\norm{x}} = \inf_{x : f(x) = 1} \norm{x}\]
    where the last inequality follows from the fact that finding the least upper bound of $1/\norm{x}$ is the same as finding the lowest upper bound of $\norm{x}$.
\end{proof}

\begin{question}
    Let $f \neq 0$ be a bounded linear functional on a real normed space $X$. Then, for any scalar $c$, we have a hyperplane $H_c = \{x \in X \mid f(x) = c\}$ and $H_c$ determines the two half spaces:
    \[X_{c1} = \{x \mid f(x) \leq c\} \;,\; X_{c2} = \{x \in X \mid f(x) \geq c\}.\]
    Show that the closed unit ball lies in $X_{c1}$ where $c = \norm{f}$, but for no $\epsilon \g 0$, the half space $X_{c1}$ with $c = \norm{f} - \epsilon$ contains that ball.
    \label{section2.8-15}
\end{question}
\begin{proof}
    The closed unit ball is the set of all points $\tilde{B}(0,1) = \{x \mid \norm{x} \leq 1\}.$ Thus, for any $x \in \tilde{B}(0,1)$, we have
    \[f(x) \leq \norm{f}\norm{x} \leq \norm{f}\]
    and hence, $\tilde{B}(0,1) \subset X_{c1}$ where $c = \norm{f}$.

    Now, we also know that
    \[\norm{f} = \sup_{\norm{x} = 1} \abs{f(x)}.\]
    Now, from the definition of a supremum, we know that for any $\epsilon \g 0$, there exists some $x \in \tilde{B}(0,1)$ such that $f(x) \g \norm{f} - \epsilon$. Hence, $\tilde{B}(0,1)$ is no longer a subset of $X_{c1}$ for $c = \norm{f} - \epsilon$.
\end{proof}